\PassOptionsToPackage{numbers}{natbib} % passing an option to  natbib before tufte-book
\documentclass[justified,openany,nofonts]{tufte-book}

\bibliographystyle{amsplain}
\usepackage{teacherEd}


%\usepackage{microtype}

%\usepackage{lineno}
%\linenumbers





%\usepackage{framed}
\usepackage{comment}

%\newenvironment{teachingnote}{\begin{framed}\begin{itshape}\begin{bfseries}\noindent Teaching Note:}{\end{bfseries}\end{itshape}\end{framed}}
%\newcommand{\teachingnotes}{With Teaching Notes}
\excludecomment{teachingnote}
\newcommand{\teachingnotes}{}



%%% This sets how the enumerate command works
\renewcommand{\theenumi}{$(\mathrm{\arabic{enumi}})$}
\renewcommand{\labelenumi}{\theenumi}


\title{Numbers \\ and Algebra}
\author{\teachingnotes}
\publisher{Fall 2013 \\ This document was typeset on \today.}


\begin{document}
\def\document#1{} %% Needed to add standards
\maketitle
%%%% Front matter
\newpage

\begin{fullwidth}
~\vfill
\thispagestyle{empty}
\setlength{\parindent}{0pt}
\setlength{\parskip}{\baselineskip}
Copyright \copyright~2025 Bart Snapp, Victor Ferdinand, Bradford Findell, and Betsy McNeal

\vspace{.5cm}

\noindent
This work is licensed under the Creative Commons:
\begin{center}
Attribution-NonCommercial-ShareAlike License 
\end{center}
To view a copy of this license, visit \url{http://creativecommons.org/licenses/by-nc-sa/3.0/}.

\vspace{.5cm}
\noindent This document was typeset on \today.
\end{fullwidth}


\chapter*{Preface}
\addcontentsline{toc}{chapter}{Preface}


These notes are designed with future middle grades mathematics
teachers in mind.  While most of the material in these notes would be
accessible to an accelerated middle grades student, it is our hope
that the reader will find these notes both interesting and
challenging.  In some sense we are simply taking the topics from a
middle grades class and pushing ``slightly beyond'' what one might
typically see in schools. In particular, there is an emphasis on the
ability to communicate mathematical ideas.  Three goals of these notes
are:
\begin{itemize}
\item To enrich the reader's understanding of both numbers and algebra. 
From the basic algorithms of arithmetic---all of which have algebraic
underpinnings---to the existence of irrational numbers, we hope to show
the reader that numbers and algebra are deeply connected.
\item To place an emphasis on problem solving. The reader will be exposed 
to problems that ``fight-back.'' Worthy minds such as yours deserve
worthy opponents. Too often mathematics problems fall after a single
``trick.'' Some worthwhile problems take time to solve and cannot be done
in a single sitting.
\item To challenge the common view that mathematics is a body of knowledge 
to be memorized and repeated. The art and science of doing mathematics
is a process of reasoning and personal discovery followed by
justification and explanation. We wish to convey this to the reader,
and sincerely hope that the reader will pass this on to others as
well.
\end{itemize}
In summary---you, the reader, must become a doer of mathematics.  To
this end, many questions are asked in the text that follows. Sometimes
these questions are answered; other times the questions are left for
the reader to ponder. To let the reader know which questions are left
for cogitation, a large question mark is displayed:
\QM
The instructor of the course will address some of these questions. If
a question is not discussed to the reader's satisfaction, then we
encourage the reader to put on a thinking-cap and think, think, think!
If the question is still unresolved, go to the World Wide Web and
search, search, search!

Much of the mathematics content in this course is 
strongly tied to the mathematics that you may be teaching in grades 4
through 9.  To emphasize these connections, you will sometimes
see margin notes that begin ``CCSS.''  These are drawn
from the \textit{Common Core State Standards}, which describe goals for 
mathematics learning in grades K--12 in Ohio and many other states.  
For more information, see \url{http://www.corestandards.org}.  

This document is open-source. It is licensed under the Creative
Commons Attribution-NonCommercial-ShareAlike (CC BY-NC-SA)
License. Loosely speaking, this means that this document is available
for free. Anyone can get a free copy of this document 
from the following sites:
\begin{center}
\url{http://www.math.osu.edu/~snapp/1165/}

\url{http://www.math.osu.edu/~findell.2}
\end{center}

Please report corrections, suggestions, gripes, complaints, and
criticisms to Bart Snapp at \href{mailto:snapp@math.osu.edu}{snapp@math.osu.edu} or Brad Findell
at \href{mailto:findell.2@osu.edu}{findell.2@osu.edu}.  


\section*{Thanks and Acknowledgments}

This document is based on a set of lectures originally given by Bart
Snapp at the Ohio State University Fall 2009 and Fall 2010.  
Since then, additional text and many activities have been added by 
Vic Ferdinand, Brad Findell, and Betsy McNeal as part of our ongoing 
revision process to better serve our audience of future middle grades 
teachers. Special thanks goes to Herb Clemens for many helpful comments
that have greatly improved these notes.


\makeatletter %% adds space so that the numbers of the toc don't bump
\renewcommand{\l@section}{\@dottedtocline{1}{5em}{5em}}
\renewcommand{\l@subsection}{\@dottedtocline{2}{5em}{5em}}
\renewcommand{\l@subsubsection}{\@dottedtocline{3}{5em}{5em}}
\makeatother

\setcounter{tocdepth}{1}
\tableofcontents










%%%%
\newpage
%\pagenumbering{arabic}
%\pagestyle{fancy}
%%%%%%%%%%%%%%%%%%%%%%%%%%%%%%%%%%%%%%%%
%%%%%%% Sections to be included %%%%%%%%
%%%%%%%%%%%%%%%%%%%%%%%%%%%%%%%%%%%%%%%%
\setcounter{secnumdepth}{2}% turn on numbering for parts and chapters

Chapters go here.  



\appendix

\renewcommand{\theenumi}{$(\mathrm{\alph{enumi}})$}
\renewcommand{\labelenumi}{\theenumi}
\chapter{Activities}
%\addtocontents{toc}{\protect\setcounter{tocdepth}{0}}


\section{The Triathlete}\label{A:Triathlete}

\begin{prob} 
On Friday afternoon, just as Laine got off the bus, she realized that she had left her bicycle at school.  In order to have her bicycle at home for the weekend, she decided to run to school and then ride her bike back home.  If she averaged 6 mph running and 12 mph on her bike, what was her average speed for the round trip?  Explain your reasoning. 
\end{prob}
\begin{prob}
On Saturday, Laine completed a workout in which she split the time evenly between running and cycling.  If she again averaged 6 mph running and 12 mph on her bike, what was her average speed for the workout?  Explain your reasoning. 
\end{prob}
\begin{prob}
Why was her average speed on Saturday different from her average speed on Friday?  Can you reason, without computation, which average speed should be faster?  
\end{prob}
\begin{prob} On Sunday, Laine's workout included swimming.  Assuming that she can swim at an average speed of 2 mph, describe two running-cycling-swimming workouts, one similar to Friday's scenario (same distance) and a second similar to Saturday's (same time).  Compute the average speed for each and explain your reasoning.  
\end{prob}
\begin{prob}
Which of the workout scenarios (same distance or same time) most closely resembles an actual triathlon?  Why do you think that is the case?
\end{prob}
\begin{prob}
After two months of intense training, Laine is able to average $s$ mph swimming, $r$ mph running, and $c$ mph cycling.  Again describe two running-cycling-swimming workouts, one similar to each of the two original scenarios, and compute her average speeds.     
\end{prob}

\newpage
\section{Constant Amount Changes}\label{A:ConstantAmount}

\begin{prob}
Gertrude the Gumchewer has an addiction to Xtra Sugarloaded Gum, and it's getting worse.  Each day, she goes to her always stocked storage vault and grabs gum to chew.  At the beginning of her habit, she chewed three pieces and then, each day, she chews eight more pieces than she chewed the day before to satisfy her ever-increasing cravings.
\begin{enumerate}
\item Gertrude's friend Wanda has been keeping tabs on Gertrude's habit.  She notices that Gertrude chewed 35 pieces on day 5.  Wanda claims that, because Gertrude is increasing the number of pieces she chews at a constant rate, we can just use proportions with the given piece of information to find out how many pieces Gertrude chewed on any other day.  Is Wanda correct or not?

\item Make a table of how pieces of gum Gertrude chewed on each of the first 10 days of her addiction.  Be sure to show the arithmetic process you go through for each day (i.e., not just the final number of pieces).  Find a pattern that will predict an answer.  

\item How many pieces of gum Gertrude did chew on the $793^\mathrm{rd}$ day of her habit?  How many pieces did she chew on the $n^\mathrm{th}$ day of her habit?  Explain your reasoning.  

\item Think of what a $4^\mathrm{th}$ grader would do to predict the next day's number of pieces given the previous day's number of pieces.  How would the $4^\mathrm{th}$ grader answer the previous question?  How does this differ from how you solved it?

\item What you (likely) did in the previous part is called a ``recursive'' representation of the relationship: Finding the next value from previous values.  We can use function notation for sequences, with $f(n)$ representing the $n^\mathrm{th}$ term of a sequence.  Many recursive relationships can be specified by initial condition and a general (recursion) formula, such as the following:     
\begin{itemize}
\item Initial Condition: $f(0) = 3$.
\item General Term: $f(n+1) = f(n) + 8$.   
\end{itemize}
Note that the initial condition is crucial, because if we changed the initial chewing to 5 pieces, all the numbers would be different.   Find $f(1)$ through $f(5)$.  Where do you see these values in your previous work?  
\item Make a graph of your data about Getrude's gum chewing.  Which variable do you plot on the horizontal axis?  Explain.  
\item Does it make sense to connect the dots on your graph?  Explain your reasoning.  
\item  Using your table from above, compute the differences between the number of pieces chewed on successive days (e.g.,  $f(1) - f(0)$, $f(2) - f(1)$, etc.).  What do you notice?  Why does this happen?  
\end{enumerate}
\end{prob}

\begin{prob}
Slimy Sam steals a car from a rest area 3 miles east of the Indiana-Ohio state line and starts heading east along the side of I-70.  Because the car is a real clunker, it can only go 8 miles per hour.  
\begin{enumerate}
\item Assuming the police are laughing too hard to arrest Sam, describe Sam's position on I-70 (via mile markers) $x$ hours after stealing the car.  
\item Make a graph of your data about Sam's travel.  Which variable do you plot on the horizontal axis?  Explain.  
\item Does it make sense to connect the dots on your graph?  Explain your reasoning.  
\item How is this problem fundamentally different from the Gertrude problems?  
\item Dumb Question:  At any specific time, how many positions could Sam be in? 
\end{enumerate}
\end{prob}

\newpage
\section{Constant Percentage (Ratio) Changes}\label{A:ConstantRatio}

\begin{prob}
Billy is a bouncing ball.  He is dropped from a height of 13 feet and each bounce goes up 92\% of the bounce before it.  Assume that the first time Billy hits the ground is bounce 1.  

\begin{enumerate}
\item Make a table of how high Billy bounced after each of the first 10 times he hit the ground.  Be sure to indicate the arithmetic process you go through for each bounce (i.e., not just the final height).  Find a pattern that will predict an answer.  

\item How high will Billy bounce after the 38th bounce?  How high will Billy bounce after the $n^\mathrm{th}$ bounce?  Explain your reasoning. 

\item Think of what a $4^\mathrm{th}$ grader would do to predict the next bounce's height given the previous bounce's height.  How would the $4^\mathrm{th}$ grader answer the previous question?  How does this differ from how you solved it?

\item Make a graph of your data about Billy.  Which variable do you plot on the horizontal axis?  Explain.  
\item Does it make sense to connect the dots on your graph?  Explain your reasoning.  

\item  Use function notation to specify the height of Billy's bounces as a recursive sequence, including the initial condition and general term.   Use this notation to find $f(1)$ through $f(5)$.

\item Using your table from above, compute the differences between the heights on successive bounces (e.g.,  $f(1) - f(0)$, $f(2) - f(1)$, etc.).  What do you notice?  Why does this happen?

\item Compare and contrast the explicit and recursive representations from Billy and from Gertrude.  How do the role(s) of the operations and initial values differ, remain the same, or relate?
\end{enumerate}
\end{prob}

\begin{prob}
Supppose 5 mg of a drug is administered to a patient once, and the amount of the drug in the patient's body decreases by 21\% each day.  
\begin{enumerate}
\item Describe the amount of the drug in the patient's body $x$ days after it was administered.  
\item Make a graph of your data about the amount of drug in the body over time.  Which variable do you plot on the horizontal axis?  Explain.  
\item Does it make sense to connect the dots on your graph?  Explain your reasoning.  

\item How is this problem fundamentally different from the Billy problems?  
\item Dumb Question:  At any one time, how many different amounts of the drug are possible in the patient's body?
\end{enumerate}
\end{prob}

\newpage
\section{Rules of Exponents}\label{A:ExponentRules}

\begin{prob}
Attendance at a picnic consistently grows by 21\% each year.  The attendance this year was 5678.    
\begin{enumerate}
\item Write both a recursive and explicit representation of the relationship between the number of years and the attendance.  Use 2013 as ``year 0.''

\item Using only the attendance for 2018, how would you find the attendance for 2024 by doing only one multiplication?  What rule from school mathematics supports your solution process?
      
\item Using only the explicit form of the relationship, predict what the attendance was in 2010.  Now find the same information by only using the attendance in 2018 and one division. What rule from school mathematics supports your solution process?
     
\item What was the attendance in 2013?  What rule from school mathematics supports your solution process?
\end{enumerate}
\end{prob}
\begin{prob}
What is wrong with the following statement:  ``$5^7$ is 5 multiplied by itself 7 times.''  If the statement were true, what would $5^1$ be?  What would $5^0$ be?
\end{prob}
\begin{prob}
Why is $x^3$ not the same function as $3^x$?  We often think of multiplication as ``repeated addition,'' and we find that adding $a$ copies of $b$ gives the same result as adding $b$ copies of $a$.  Does this idea work for thinking of exponentiation as ``repeated multilpication''?  Explain.  
\end{prob}

\begin{prob}
Joe Schmo saved the King's daughter from a vicious dragon.  For such gallantry, the King offers Joe the choice of two payment plans: 
\begin{itemize}
\item Plan \#1:  \$1 today, and on subsequent days Joe will have an amount that is the cube of the day number (so, for example, on day 2, Joe  will have \$8). 
\item Plan \#2:  \$1 today, and on subsequent days Joe will have 2\% more than the previous day (for example, on Day \#2, he'll have \$1.02).  
\end{itemize}
Note that Joe must deposit the previous day's money in order to get today's money.  Assuming he wants the plan that yields the most money, which plan should he pick?  Explain your reasoning.  
\end{prob}

\section{Garden Variety}\label{A:GardenVariety}
\begin{prob}
A park consists of a row of circular gardens.  ``Garden \#0'' has radius 3 feet, and each successive garden after that has a radius 2 feet greater than the previous garden.  
\begin{enumerate}
\item Using tables as a guide, write both explicit and recursive representations that will allow us to predict the area of the $n^\mathrm{th}$ garden.
\item Make a graph 
\item Make a graph that shows the areas of the gardens in the park.  Which variable do you plot on the horizontal axis?  Explain.  
\item Does it make sense to connect the dots on your graph?  Explain your reasoning.  
\item Using your table, compute the area differences between the successive gardens.  What do you notice?  Why does this happen?
\end{enumerate}
\end{prob}
\begin{prob}
An oil spill starts out as a circle with radius 3 feet and is expanding outward in all directions at a rate of 2 feet per minute. 
\begin{enumerate}
\item Use tables, graphs, and formulas to describe the area of the oil region $x$ minutes after the spill.  
\item How is this question fundamentally different than that of the gardens?  
\item Dumb Question:  At any one time, how many different areas are possible for the oil region?

\end{enumerate}
\end{prob}

\newpage
\section{The World Series}\label{A:WorldSeries}

\begin{prob}
Recall the story of Gertrude the Gumchewer, who has an addiction to Xtra Sugarloaded Gum.  Each day, she goes to her always stocked storage vault and grabs gum to chew.  At the beginning of her habit, she chewed three pieces and then, each day, she chews eight more pieces than she chewed the day before to satisfy her ever-increasing cravings. We want to find out how many pieces of gum did Gertrude chew over the course of the first 973 days of her habit?

\end{prob}

\begin{prob}\label{P:gtg2}
Assume now that Gertrude, at the beginning of her habit, chewed $m$
pieces of gum and then, each day, she chews $n$ more pieces than she
chewed the day before to satisfy her ever-increasing cravings.  How many pieces will she chew over the course of the first $k$
  days of her habit? Explain your formula and how you know it will work for any $m$, $n$ and $k$.  
\end{prob}

\begin{prob}
Use the method you developed in questions \ref{P:gtg1} and
\ref{P:gtg2} to find the sum:
\[
19 + 26 + 33 + \dots + 1720
\]
Give a story problem that is represented by this sum.
\end{prob}

\begin{prob}
Now remember the story of Billy the bouncing ball.  He is dropped from a height of 13 feet and each bounce goes up 92\% of the bounce before it.  Assume that the first time Billy hits the ground is bounce \#1.  How far did Billy travel over the course of 38 bounces (up to when he hits the ground on his 38th bounce)?  
\end{prob}

\begin{prob}
Assume now that Billy the Bouncing Ball is dropped from a height of
$h$ feet. After each bounce, Billy goes up a distance equal to $r$
times the distance of the previous bounce. (For example, $r=.92$ above.)
\begin{enumerate}
\item How high will Billy go after the $k$th bounce?
\item How much distance will Billy travel over the course of $k$
  bounces (not including the height he went up after the $k$th
  bounce)?
\item If $r<1$, what can you say about Billy's bounces? What if $r=1$?
  What if $r>1$?
\end{enumerate}
\end{prob}

\end{document}
