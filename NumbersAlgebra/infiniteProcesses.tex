\documentclass[nooutcomes]{ximera}

\title{A Note on Infinite Process}
\author{Bradford Findell, Ohio State University}
\begin{document}
\begin{abstract}
Here we describe ways of thinking about infinite processes.  
\end{abstract}
\maketitle

Mathematical reasoning often involves ``infinite processes'' in which direct calculation is impossible.\footnote{This discussion draws heavily on ideas described in \emph{Where Mathematics Comes  From: How the embodied mind brings mathematics into being}  by Lakoff and N\'{u}\~{n}ez (2000).}  Infinite processes become central in calculus, where both differentiation and integration are defined via limits.  These approaches are made rigorous in advanced undergraduate courses, such as Real Analysis.  But infinite processes arise from time to time even in middle grades mathematics, and so it is important that teachers are able to talk about them sensibly and accurately.  Here we explain some key ideas for reasoning about infinite processes.  

First, there is the idea of a process that continues, over and over, without end.  
Here are some examples:
\begin{quote}
\begin{itemize}
\item Perhaps the earliest of these is counting: 1, 2, 3, 4, \dots.  We do not imagine completing the process of counting.  Nonetheless, for any large positive number you name, we can imagine exceeding that number, eventually, if we have enough time.  
\item We can approximate $1/3$ with a sequence, $0.3$, $0.33$, $0.333$, and so on.  We can get as close to $1/3$ as we like by including enough digits.  Note, on the other hand, that it is false to say that $0.3333 = 1/3$ or even $0.3333333333 = 1/3$, because any finite number of digits will miss $1/3$ by an amount that can be calculated precisely.  
\item If we look at a sequence of regular $n$-gons of the same diameter, as $n$ gets large, we can get as close to a circle as we might like.  But for any finite number of sides, the regular $n$-gon will not actually be a circle.  
\end{itemize}
\end{quote}

The above examples use what is sometimes called \emph{potential infinity}, for in none of the cases do we actually complete the process, and we do not need to.  We imagine these things as going on ``forever,'' and a process that goes on forever never ends.  

But the interesting uses of infinity in mathematics involve \emph{actual infinity}.  
\begin{question}
In each of the above examples, what would happen if the process could end?
\end{question}
%\QM

In order to conceptualize actual infinity, we imagine, metaphorically, that the process \emph{does} end.  In a literal sense, an infinite process cannot end, but through the use of metaphor, we consider what would happen if the process were to end.  And with the help of intuitions about completed processes, we then infer the ``ultimate result'' of the completed infinite process.    

With the metaphor of actual infinity, counting yields the infinite set of counting numbers, $\mathbb{N}$.  All of them.  In the repeating decimal for $1/3$, we get an exact decimal representation, so that $0.33333\ldots = 1/3$.  Exactly.  And in the case of the regular $n$-gon with an infinite number of sides, we get a circle.  Perfectly.  

\begin{warning}
With the metaphor of actual infinity, it is \emph{false} to say that $0.3333\ldots$ never gets to $1/3$ because the dots imply that the infinite process has been completed.  Although any finite number of digits fails reach $1/3$, an infinite number of digits reaches $1/3$ exactly:  The error has gone to 0.  
\end{warning}

In summary, reasoning about infinite processes involves the following steps:  
\begin{quote}
\begin{enumerate}
\item Describing the finite process carefully and accurately;
\item Considering the process to go on forever, and describing how the result can get arbitrarily close to some goal;
\item Imagining that the infinite process has been completed; and
\item Reasoning about the ``ultimate result'' of the infinite process.
\end{enumerate}
\end{quote}

For some infinite processes, it is quite helpful in the second and fourth steps to talk about the ``error,'' which is to say how much the finite process falls short of the ultimate goal, and then to argue that the error becomes arbitrarily small (i.e., it goes to 0). 

\vspace{0.2in}

Happy infinite reasoning!   
\end{document}

