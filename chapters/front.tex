

\newpage

\begin{fullwidth}
~\vfill
\thispagestyle{empty}
\setlength{\parindent}{0pt}
\setlength{\parskip}{\baselineskip}
Copyright \copyright~2017 Bart Snapp, Victor Ferdinand, Bradford Findell, and Betsy McNeal

\vspace{.5cm}

\noindent
This work is licensed under the Creative Commons:
\begin{center}
Attribution-NonCommercial-ShareAlike License 
\end{center}
To view a copy of this license, visit \url{http://creativecommons.org/licenses/by-nc-sa/3.0/}.


\vspace{.5cm}
\noindent This document was typeset on \today.
\end{fullwidth}



\chapter*{Preface}
\addcontentsline{toc}{chapter}{Preface}


These notes are designed with future middle grades mathematics
teachers in mind.  While most of the material in these notes would be
accessible to an accelerated middle grades student, it is our hope
that the reader will find these notes both interesting and
challenging.  In some sense we are simply taking the topics from a
middle grades class and pushing ``slightly beyond'' what one might
typically see in schools. In particular, there is an emphasis on the
ability to communicate mathematical ideas.  Three goals of these notes
are:
\begin{itemize}
\item To enrich the reader's understanding of both numbers and algebra. 
From the basic algorithms of arithmetic---all of which have algebraic
underpinnings---to the existence of irrational numbers, we hope to show
the reader that numbers and algebra are deeply connected.
\item To place an emphasis on problem solving. The reader will be exposed 
to problems that ``fight-back.'' Worthy minds such as yours deserve
worthy opponents. Too often mathematics problems fall after a single
``trick.'' Some worthwhile problems take time to solve and cannot be done
in a single sitting.
\item To challenge the common view that mathematics is a body of knowledge 
to be memorized and repeated. The art and science of doing mathematics
is a process of reasoning and personal discovery followed by
justification and explanation. We wish to convey this to the reader,
and sincerely hope that the reader will pass this on to others as
well.
\end{itemize}
In summary---you, the reader, must become a doer of mathematics.  To
this end, many questions are asked in the text that follows. Sometimes
these questions are answered; other times the questions are left for
the reader to ponder. To let the reader know which questions are left
for cogitation, a large question mark is displayed:
\QM
The instructor of the course will address some of these questions. If
a question is not discussed to the reader's satisfaction, then we
encourage the reader to put on a thinking-cap and think, think, think!
If the question is still unresolved, go to the World Wide Web and
search, search, search!

Much of the mathematics content in this course is 
strongly tied to the mathematics that you may be teaching in grades 4
through 9.  To emphasize these connections, you will sometimes
see margin notes that begin ``CCSS.''  These are drawn
from the \textit{Common Core State Standards}, which describe goals for 
mathematics learning in grades K--12 in Ohio and many other states.  
For more information, see \url{http://www.corestandards.org}.  

This document is open-source. It is licensed under the Creative
Commons Attribution-NonCommercial-ShareAlike (CC BY-NC-SA)
License. Loosely speaking, this means that this document is available
for free. Anyone can get a free copy of this document 
from the following sites:
\begin{center}
\url{http://www.math.osu.edu/~snapp/1165/}

\url{http://www.math.osu.edu/~findell.2}
\end{center}

Please report corrections, suggestions, gripes, complaints, and
criticisms to Bart Snapp at \href{mailto:snapp@math.osu.edu}{snapp@math.osu.edu} or Brad Findell
at \href{mailto:findell.2@osu.edu}{findell.2@osu.edu}.  


\section*{Thanks and Acknowledgments}

This document is based on a set of lectures originally given by Bart
Snapp at the Ohio State University Fall 2009 and Fall 2010.  
Since then, additional text and many activities have been added by 
Vic Ferdinand, Brad Findell, and Betsy McNeal as part of our ongoing 
revision process to better serve our audience of future middle grades 
teachers. Special thanks goes to Herb Clemens for many helpful comments
that have greatly improved these notes.


\makeatletter %% adds space so that the numbers of the toc don't bump
\renewcommand{\l@section}{\@dottedtocline{1}{5em}{5em}}
\renewcommand{\l@subsection}{\@dottedtocline{2}{5em}{5em}}
\renewcommand{\l@subsubsection}{\@dottedtocline{3}{5em}{5em}}
\makeatother

\setcounter{tocdepth}{1}
\tableofcontents









