\section{The Adders}



If long division is a \textit{forgotten foe}\index{forgotten foes},
then logarithms\index{logarithm} are a
\textit{supervillan}\index{supervillan}. When aloof old Professor
Rufus was trying to explain logarithms to his class, he merely wrote
\[
\log_b(a) = n \qquad\Leftrightarrow\qquad b^n = a
\]
and walked out of the room.

\begin{question}
Can you give $3$ much needed examples of logarithms that are easily
computed in one's head?
\end{question}
\QM

\begin{question}
Can you give $3$ much needed examples of logarithms that are more
difficult to compute in one's head?
\end{question}
\QM

\begin{question} 
What conditions should be placed on $a$ and $b$ to make logarithms
work nicely?
\end{question}
\QM

\begin{question} What is $\log_b(1)$?
\end{question}
\QM

\begin{question} What is $\log_b(0)$?
\end{question}
\QM


\begin{question} What is $\log_1(1)$?
\end{question}
\QM

\begin{question} Sketch the plot of $y = \log_b(x)$. 
\end{question}
\QM


\begin{question} Why is this section named ``interlude of the adders?!''
\end{question}
\QM




\begin{problems}
\begin{enumerate}
\item Explain what $\log_b(a) = n$ means.
\item Sketch the plot of $y=\log_b(x)$ for some reasonable value of
  $b$. Explain your procedure.
\item Sketch the plot of $y=b^x$ for some reasonable value of
  $b$. Explain your procedure. How does this plot compare to the one
  in the previous question?
\item What is $\log_x(x^3)$? Explain your reasoning.
\item Given that $\ln(x) = \log_e(x)$, explain why is it no big deal
  to say that $\ln(e^x) = x$.
\item Compute $\log_{5}(125)$. Explain your reasoning.
\item Compute $\log_{10}(10000)$. Explain your reasoning.
\item Compute $\log_2(1024)$. Explain your reasoning.
\item Compute $\log_{13}(169)$. Explain your reasoning.
\item Compute $\log_{7}(2401)$. Explain your reasoning.
\item Bound $\log_{2}(5)$ by two consecutive integers. Explain your reasoning.
\item Bound $\log_{3}(43)$ by two consecutive integers. Explain your reasoning.
\item Bound $\log_{11}(24)$ by two consecutive integers. Explain your
  reasoning.
\item Bound $\log_{10}(999)$ by two consecutive integers. Explain your
  reasoning.
\item Bound $\log_{10}(1032)$ by two consecutive integers. Explain
  your reasoning.
\item What is the connection between the number of digits in some
  number $n$ and $\log_{10}(n)$? Explain your reasoning.
\item How many digits does the number $100$ have in base $2$? What
  does this have to do with $\log_2(100)$? Explain your reasoning.
\item How many digits does the number $100$ have in base $3$? What
  does this have to do with $\log_3(100)$? Explain your reasoning.
\item How many digits does the number $100$ have in base $11$? What
  does this have to do with $\log_{11}(100)$? Explain your reasoning.
\item How many digits does the number $100$ have in base $42$? What
  does this have to do with $\log_{42}(100)$? Explain your reasoning.
\item How many digits does the number $100$ have in base $99$?  What
  does this have to do with $\log_{99}(100)$? Explain your reasoning.
\item How many digits does the number $100$ have in base $100$?  What
  does this have to do with $\log_{100}(100)$? Explain your reasoning.
\item How many digits does the number $100$ have in base $101$?  What
  does this have to do with $\log_{101}(100)$? Explain your reasoning.
\item Explain why $\log_b(a) + \log_b(c) = \log_b(a\cdot c)$.
\item Explain why $\log_b(a) - \log_b(c) = \log_b(a/c)$.
\item Explain why $c\cdot\log_b(a) = \log_b(a^c)$.
\item Explain why $\log_b(a) = \dfrac{1}{\log_a(b)}$.
\item Explain why $\log_b(a) = \dfrac{\log_c(a)}{\log_c(b)}$.
\item People have often told me something like ``it is impossible to
  fold a piece of paper more than $7$ times.'' What is meant by this
  statement and is it even true? Explain your reasoning. Note if you
  cannot solve this problem, no worries just say to yourself (aloud so
  all can hear) ``I believe you can fold a piece of paper as much as
  you want'' three times and then do the next problem and then come
  back to this one.
\item Take a sheet of paper. If you fold it once, the resulting folded
  sheet of paper is twice as thick as the unfolded paper. If you fold
  it again, the resulting folded sheet is 4 times as thick as the
  unfolded piece of paper. How many times would you need to fold a
  sheet of paper to make the resulting sheet of paper as thick as you
  are tall? Explain your reasoning. Don't bother worrying about the
  physical limitations of this problem.
\item Explain why the following ``joke'' is ``funny:'' 
\begin{quote} 
The Flood is over and the ark has landed. Noah lets all the animals
out and says, ``Go forth and multiply.''

A few months later, Noah decides to take a stroll and see how the
animals are doing. Everywhere he looks he finds baby animals. Everyone
is doing fine except for one pair of little snakes. ``What's the
problem?'' says Noah.  ``Cut down sssome treesss and let usss live there,'' say
the snakes.

Noah follows their advice. Several more weeks pass. Noah checks on the
snakes again. Lots of little snakes, everybody is happy. Noah asks,
``Want to tell me how the trees helped?''

``SSSertainly,'' say the snakes. ``We're addersss, ssso we need logsss
to multiply.''
\end{quote}
%%
%% Ideas for new questions
%% *) Mult by table
%% 
\end{enumerate}
\end{problems}
