\newpage
\section{Garden Variety}\label{A:gardenVariety}
\fixnote{Move these to problems.  Use board pictures on sequences, functions, quadratics, to create new activity.}

\begin{prob}
A park consists of a row of circular gardens.  ``Garden \#0'' has radius 3 feet, and each successive garden after that has a radius 2 feet greater than the previous garden.  
\begin{enumerate}
\item Using tables as a guide, write both explicit and recursive representations that will allow us to predict the area of the $n^\mathrm{th}$ garden.
\item Make a graph that shows the areas of the gardens in the park.  Which variable do you plot on the horizontal axis?  Explain.  
\item Does it make sense to connect the dots on your graph?  Explain your reasoning.  
\item Using your table, compute the area differences between the successive gardens.  What do you notice?  Why does this happen?
\end{enumerate}
\end{prob}
\begin{prob}
An oil spill starts out as a circle with radius 3 feet and is expanding outward in all directions at a rate of 2 feet per minute. 
\begin{enumerate}
\item Use tables, graphs, and formulas to describe the area of the oil region $x$ minutes after the spill.  
\item How is this question fundamentally different than that of the gardens?  
\item Dumb Question:  At any one time, how many different areas are possible for the oil region?

\end{enumerate}
\end{prob}
