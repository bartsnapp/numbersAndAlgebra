\newpage
\section{Constant Amount Changes}\label{A:ConstantAmount}

\begin{prob}\label{P:gtg1}
Gertrude the Gumchewer has an addiction to Xtra Sugarloaded Gum, and it's getting worse.  Each day, she goes to her always stocked storage vault and grabs gum to chew.  At the beginning of her habit, she chewed three pieces and then, each day, she chews eight more pieces than she chewed the day before to satisfy her ever-increasing cravings.
\begin{enumerate}
\item Gertrude's friend Wanda has been keeping tabs on Gertrude's habit.  She notices that Gertrude chewed 35 pieces on day 5.  Wanda claims that, because Gertrude is increasing the number of pieces she chews at a constant rate, we can just use proportions with the given piece of information to find out how many pieces Gertrude chewed on any other day.  Is Wanda correct or not?

\item Make a table of how pieces of gum Gertrude chewed on each of the first 10 days of her addiction.  Be sure to show the arithmetic process you go through for each day (i.e., not just the final number of pieces).  Find a pattern that will predict an answer.  

\item How many pieces of gum Gertrude did chew on the $793^\mathrm{rd}$ day of her habit?  How many pieces did she chew on the $n^\mathrm{th}$ day of her habit?  Explain your reasoning.  

\item Think of what a $4^\mathrm{th}$ grader would do to predict the next day's number of pieces given the previous day's number of pieces.  How would the $4^\mathrm{th}$ grader answer the previous question?  How does this differ from how you solved it?

\item What you (likely) did in the previous part is called a ``recursive'' representation of the relationship: Finding the next value from previous values.  We can use function notation for sequences, with $f(n)$ representing the $n^\mathrm{th}$ term of a sequence.  Many recursive relationships can be specified by initial condition and a general (recursion) formula, such as the following:     
\begin{itemize}
\item Initial Condition: $f(0) = 3$.
\item General Term: $f(n+1) = f(n) + 8$.   
\end{itemize}
Note that the initial condition is crucial, because if we changed the initial chewing to 5 pieces, all the numbers would be different.   Find $f(1)$ through $f(5)$.  Where do you see these values in your previous work?  
\item Make a graph of your data about Getrude's gum chewing.  Which variable do you plot on the horizontal axis?  Explain.  
\item Does it make sense to connect the dots on your graph?  Explain your reasoning.  
\item  Using your table from above, compute the differences between the number of pieces chewed on successive days (e.g.,  $f(1) - f(0)$, $f(2) - f(1)$, etc.).  What do you notice?  Why does this happen?  
\end{enumerate}
\end{prob}

\begin{prob}
Slimy Sam steals a car from a rest area 3 miles east of the Indiana-Ohio state line and starts heading east along the side of I-70.  Because the car is a real clunker, it can only go 8 miles per hour.  
\begin{enumerate}
\item Assuming the police are laughing too hard to arrest Sam, describe Sam's position on I-70 (via mile markers) $x$ hours after stealing the car.  
\item Make a graph of your data about Sam's travel.  Which variable do you plot on the horizontal axis?  Explain.  
\item Does it make sense to connect the dots on your graph?  Explain your reasoning.  
\item How is this problem fundamentally different from the Gertrude problems?  
\item Dumb Question:  At any specific time, how many positions could Sam be in? 
\end{enumerate}
\end{prob}
