\newpage
\section{The World Series}\label{A:WorldSeries}

\begin{prob}
Recall the story of Gertrude the Gumchewer, who has an addiction to Xtra Sugarloaded Gum.  Each day, she goes to her always stocked storage vault and grabs gum to chew.  At the beginning of her habit, she chewed three pieces and then, each day, she chews eight more pieces than she chewed the day before to satisfy her ever-increasing cravings. We want to find out how many pieces of gum did Gertrude chew over the course of the first 973 days of her habit?

\end{prob}

\begin{prob}\label{P:gtg2}
Assume now that Gertrude, at the beginning of her habit, chewed $m$
pieces of gum and then, each day, she chews $n$ more pieces than she
chewed the day before to satisfy her ever-increasing cravings.  How many pieces will she chew over the course of the first $k$
  days of her habit? Explain your formula and how you know it will work for any $m$, $n$ and $k$.  
\end{prob}

\begin{prob}
Use the method you developed in questions \ref{P:gtg1} and
\ref{P:gtg2} to find the sum:
\[
19 + 26 + 33 + \dots + 1720
\]
Give a story problem that is represented by this sum.
\end{prob}

\begin{prob}
Now remember the story of Billy the bouncing ball.  He is dropped from a height of 13 feet and each bounce goes up 92\% of the bounce before it.  Assume that the first time Billy hits the ground is bounce \#1.  How far did Billy travel over the course of 38 bounces (up to when he hits the ground on his 38th bounce)?  
\end{prob}

\begin{prob}
Assume now that Billy the Bouncing Ball is dropped from a height of
$h$ feet. After each bounce, Billy goes up a distance equal to $r$
times the distance of the previous bounce. (For example, $r=.92$ above.)
\begin{enumerate}
\item How high will Billy go after the $k$th bounce?
\item How much distance will Billy travel over the course of $k$
  bounces (not including the height he went up after the $k$th
  bounce)?
\item If $r<1$, what can you say about Billy's bounces? What if $r=1$?
  What if $r>1$?
\end{enumerate}
\end{prob}
