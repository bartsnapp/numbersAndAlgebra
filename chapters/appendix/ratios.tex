\newpage
\section{Poor Old Horatio}\label{A:ratios}
In this activity we are going to investigate ratios, which are the cornerstone of middle school mathematics.  

We would like to avoid procedural approaches, such as, ``set up a proportion and cross multiply.''  Instead, we want to reason from the context and use pictures and tables to support the reasoning.  In cases where you ``set up an equation and solve,'' explain the reasoning behind the equation and the steps in your solution.  


\begin{prob}
A shade of orange is made by mixing 3 parts red paint with 5 parts
yellow paint.  Sam says we can add 4 cups of each color of paint and
maintain the same color.  Fred says we can quadruple both 3 and 5 and
get the same color.  
\begin{enumerate}
\item Who (if either or both) is correct?  Explain your reasoning.
\vspace{0.25in}

\item Use a table like the one below to show other paint mixtures that are the desired shade of orange.  
\[{\renewcommand{\arraystretch}{2}
\begin{tabular}{|c||c|c|c|c|c|c|c|c|c|}\hline
Red  &  3 &\rule[7mm]{10mm}{0mm} & \rule[7mm]{10mm}{0mm} & \rule[7mm]{10mm}{0mm}  & \rule[7mm]{10mm}{0mm}
 & \rule[7mm]{10mm}{0mm} & \rule[7mm]{10mm}{0mm} & \rule[7mm]{10mm}{0mm} & \rule[7mm]{10mm}{0mm}   \\ \hline
Yellow & 5 &  &  &  & & & & & \\ \hline
\end{tabular}}
\]
\end{enumerate}
\end{prob}


\begin{prob}
If we wanted to make the same orange paint but could only use 73 cups of yellow paint, how
many cups of red paint would we need?  Give a detailed explanation of your solution. And 
explain what the units are for each value in your equation.
\end{prob}

\vspace{0.25in}

\begin{prob}
If we wanted to make the same orange paint but could only use 56 cups of red paint, how
many cups of yellow paint would we need?  Give a detailed explanation of your solution. And 
explain what the units are for each value in your equation.
\end{prob}

\vspace{0.25in}

%\begin{prob}
%Is ``cross-multiplication''a legitimate way to solve the equations
%arising from this sort of problem---be sure to think of the weird
%units that are generated by doing so.  What good is this kooky method?
%What exactly is one doing when they ``cross-multiply'' and what type
%of problem does it solve?
%\end{prob}

\begin{teachingnote}
Rather than set up a proportion and cross multiply, 
we want to encourage two approaches:  (1) writing an equation and solve; and (2) reasoning from within the table.  
But how can we tell whether a table is a ratio table?
\end{teachingnote}

\begin{prob}
To answer the question about 73 cups of yellow paint, some people use a table like the following:  
\[{\renewcommand{\arraystretch}{2}
\begin{tabular}{|c||c|c|c|c|}\hline
Red  &  3 &\rule[7mm]{10mm}{0mm} &\rule[7mm]{10mm}{0mm} & \rule[7mm]{10mm}{0mm}     \\ \hline
Yellow & 5 & 1 & 73 & $x$ \\ \hline
\end{tabular}}
\]
Fill out the remainder of the table and give a general formula for 
computing how much red paint is needed when $x$ cups of yellow 
paint is used.  Be very clear about how you 
are using the table to support your reasoning.  \margincomment{Why does it make sense 
to call this approach \textit{going through one}?}
\end{prob}

\begin{prob}
Use the following table to give a general formula for 
computing how much yellow paint is needed when $y$ cups of red 
paint is used.  Be very clear about how you 
are using the table to support your reasoning. 
\[{\renewcommand{\arraystretch}{2}
\begin{tabular}{|c||c|c|c|c|}\hline
Red  &  3 &  &  &     \\ \hline
Yellow & 5  &\rule[7mm]{10mm}{0mm} &\rule[7mm]{10mm}{0mm} & \rule[7mm]{10mm}{0mm} \\ \hline
\end{tabular}}
\]
\end{prob}

\begin{prob}
Suppose apples come in bags of 13 apples each.  
\begin{enumerate}
\item If Shel has 9 bags of apples, how many apples does Shel have total?  
\item Use a table like the one below to help Shel know how many apples are in any number of bags.  
\[{\renewcommand{\arraystretch}{2}
\begin{tabular}{|c||c|c|c|c|c|c|c|c|c|}\hline
Bags  &  \rule[7mm]{10mm}{0mm} &\rule[7mm]{10mm}{0mm} & \rule[7mm]{10mm}{0mm} & \rule[7mm]{10mm}{0mm}  & \rule[7mm]{10mm}{0mm}
 & \rule[7mm]{10mm}{0mm} & \rule[7mm]{10mm}{0mm} & \rule[7mm]{10mm}{0mm} & \rule[7mm]{10mm}{0mm}   \\ \hline
Apples &  &  &  &  & & & & & \\ \hline
\end{tabular}}
\]
\item What are the units in for each number in your calculation in part (a)?  What are the units in the answer for part (a)?  Where do you see these units in your table?  
\end{enumerate}
\end{prob}
