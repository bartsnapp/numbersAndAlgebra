\newpage
\section{Poor Old Horatio}\label{A:ratios}


%  If you find that you want to write an equation, explain why the equation makes sense, and explain the steps in your solution process.  

\begin{prob}
A shade of orange is made by mixing 3 parts red paint with 5 parts
yellow paint.  Sam says we can add 4 cups of each color of paint and
maintain the same color.  Fred says we can quadruple both 3 and 5 and
get the same color.  
\begin{enumerate}
\item Who (if either or both) is correct?  Explain your reasoning.
\vspace{0.35in}

\item Use a table like the one below to show other paint mixtures that are the desired shade of orange. 
\[{\renewcommand{\arraystretch}{2}
\begin{tabular}{|c||c|c|c|c|c|c|c|c|}\hline
Red  &  3 &\rule[7mm]{12mm}{0mm} & \rule[7mm]{12mm}{0mm} & \rule[7mm]{12mm}{0mm}  & \rule[7mm]{12mm}{0mm}
 & \rule[7mm]{12mm}{0mm} & \rule[7mm]{12mm}{0mm} & \rule[7mm]{12mm}{0mm}   \\ \hline
Yellow & 5 &  &  &  & & & & \\ \hline
           &   &  &  &  & & & & \\ \hline
\end{tabular}}
\]
\end{enumerate}
\end{prob}


\begin{prob}
If we wanted to make the same orange paint but were required to use 73 cups of yellow paint, how
many cups of red paint would we need?  Explain your reasoning.  
\vspace{0.1in} 
\[{\renewcommand{\arraystretch}{2}
\begin{tabular}{|c||c|c|c|c|c|}\hline
Red  &  3 &\rule[7mm]{12mm}{0mm} &\rule[7mm]{12mm}{0mm} & \rule[7mm]{12mm}{0mm}  & \rule[7mm]{12mm}{0mm}   \\ \hline
Yellow & 5 &  &  &  & \\ \hline
           &   &   &    &  &  \\ \hline
\end{tabular}}
\]
\vspace{0.1in} 
\end{prob}


%\vspace{0.25in}

\begin{prob}
If we wanted to make the same orange paint but were required to use 56 cups of red paint, how
many cups of yellow paint would we need?  Explain your reasoning.  
\vspace{0.1in} 
\[{\renewcommand{\arraystretch}{2}
\begin{tabular}{|c||c|c|c|c|c|}\hline
Red  &  3 &\rule[7mm]{12mm}{0mm} &\rule[7mm]{12mm}{0mm} & \rule[7mm]{12mm}{0mm}  & \rule[7mm]{12mm}{0mm}   \\ \hline
Yellow & 5 &  &  &  & \\ \hline
           &   &   &    &  &  \\ \hline
\end{tabular}}
\]
\vspace{0.1in} 
\end{prob}

%\vspace{0.25in}

%\begin{prob}
%Is ``cross-multiplication''a legitimate way to solve the equations
%arising from this sort of problem---be sure to think of the weird
%units that are generated by doing so.  What good is this kooky method?
%What exactly is one doing when they ``cross-multiply'' and what type
%of problem does it solve?
%\end{prob}

\begin{prob}
Generalize your approaches to the previous problems.  
\begin{enumerate}
\item Give a general formula for computing how much red paint
is needed when $y$ cups of yellow paint is used.   
\item Give a general formula for computing how much yellow paint
 is needed when $r$ cups of red paint is used. 
\end{enumerate}
\vspace{0.1in} 
\[{\renewcommand{\arraystretch}{2}
\begin{tabular}{|c||c|c|c|c|c|}\hline
Red  &  3 &  &  &  &   \\ \hline
Yellow & 5  &\rule[7mm]{12mm}{0mm} &\rule[7mm]{12mm}{0mm} & \rule[7mm]{12mm}{0mm} & \rule[7mm]{12mm}{0mm}  \\ \hline
          &    &  &  &   &  \\ \hline
\end{tabular}}
\]
\vspace{0.1in} 
\end{prob}

\begin{prob}
Now suppose we want to make a \textbf{different shade} of orange, this time made with $\frac{3}{4}$ cup of red paint and $\frac{2}{3}$ cup of yellow paint.  How many cups of each color do you need in order to make 15 cups of the mixture?  Use the table below.  
\vspace{0.1in} 
\[{\renewcommand{\arraystretch}{2}
\begin{tabular}{|c||c|c|c|c|c|}\hline
Red  &  $\frac{3}{4}$ &  &  &  &   \\ \hline
Yellow & $\frac{2}{3}$  &\rule[7mm]{12mm}{0mm} &\rule[7mm]{12mm}{0mm} & \rule[7mm]{12mm}{0mm} & \rule[7mm]{12mm}{0mm}  \\ \hline
          &  & $17$  & $1$ & $15$   &  \\ \hline
\end{tabular}}
\]
\vspace{0.1in} 
\end{prob}

%\margincomment{Why does it make sense 
%to call this approach \textit{going through one}?}

\fixnote{Need to distinguish ratios and rates.  Maybe earlier distinguish part/part to part/whole.}

\begin{prob}
In proportional reasoning problems, a \emph{unit rate} describes the amount of one quantity for 1 unit of another quantity.  
\begin{enumerate}
\item What are the units for the various numbers in these problems?  
\item Identify some unit rates in this activity.  
\item In solving the above problems, it is likely that you or your classmates use strategies that made use of unit rates on the way to your solution.  Explain why this strategy is sometimes called \emph{going through one}.
\end{enumerate}
\end{prob}

\fixnote{Revise these problems drawn from Beckmann.  Use dollars/pound, or meters/second, etc.}

\begin{prob}
If $2\frac{1}{2}$ pints of jelly fills $3\frac{1}{2}$ jars, then how many jars will you need for 12 pints of jelly?  (Assume the jars are all the same size.)  If the last jar is not totally full, indicate how full it will be.  
\vspace{0.1in} 
\[{\renewcommand{\arraystretch}{2}
\begin{tabular}{|c||c|c|c|c|c|c|c|c|}\hline
Jelly  &  \rule[7mm]{12mm}{0mm} &\rule[7mm]{12mm}{0mm} & \rule[7mm]{12mm}{0mm} & \rule[7mm]{12mm}{0mm}  
 & \rule[7mm]{12mm}{0mm} & \rule[7mm]{12mm}{0mm} & \rule[7mm]{12mm}{0mm} & \rule[7mm]{12mm}{0mm}   \\ \hline
Jars &  &  &  &  & & & & \\ \hline
           &  &  &  &  & & & & \\ \hline
\end{tabular}}
\]
\vspace{0.1in} 
\end{prob}


