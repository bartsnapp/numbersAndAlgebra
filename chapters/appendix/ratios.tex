\newpage
\section{Poor Old Horatio (Ratios: The Cornerstone of Middle Childhood Math)}

In this activity we are going to investigate ratios. 


\begin{prob}
A shade of orange is made by mixing 3 parts red paint with 5 parts
yellow paint.  Sam says we can add 4 cups of each color of paint and
maintain the same color.  Fred says we can quadruple both 3 and 5 and
get the same color.  Who (if either or both) is correct?  Explain your
reasoning.
\end{prob}

\begin{prob}
In keeping with the orange paint described above, if we wanted to make
the same orange paint but could only use 73 cups of yellow paint, how
many cups of red paint would we need?  Give a very detailed
explanation of your solution. In particular, if you write an equation,
you must justify why the equation holds, and explain what the units
are for each value in your equation.
\end{prob}


\begin{prob}
Consider the following table:
\[{\renewcommand{\arraystretch}{2}
\begin{tabular}{|c||c|c|c|c|}\hline
Red  &  3 &\rule[7mm]{10mm}{0mm} &\rule[7mm]{10mm}{0mm} & \rule[7mm]{10mm}{0mm}     \\ \hline
Yellow & 5 & 1 & 73 & $x$ \\ \hline
\end{tabular}}
\]
Fill out the remainder of the table. Give a general formula for
computing how much red paint is needed, and explain why this makes
sense within the context of the problem.
\end{prob}

\begin{prob}
Is ``cross-multiplication''a legitimate way to solve the equations
arising from this sort of problem---be sure to think of the weird
units that are generated by doing so.  What good is this kooky method?
What exactly is one doing when they ``cross-multiply'' and what type
of problem does it solve?
\end{prob}


\begin{prob}
Consider the following question: 
\begin{quote}
If Shel has 9 bags each with 13 apples in them, how many apples does
Shel have total?  
\end{quote}
What is somewhat unusual about the units assigned the values?  Why
does this have the potential to confuse people?  Where do we see/use
this sort of quantity in real life and throughout the standard school
curriculum?
\end{prob}
