\newpage
\section{Maximums and Minimums}\label{A:vertex}

\begin{teachingnote}
This activity will be necessary for computing least squares
approximation.
\end{teachingnote}


While you might have encountered completing the
square\index{completing the square} first when solving quadratic
equations, its real power is in transforming the form of an
expression. In this activity, we'll see it in action.\fixnote{Only real numbers for $x$.  Reason about the values.} 

\begin{prob}
Consider the curve $f(x) = x^2 -3$. Find the $x$ and $y$ values for
the maximum/minimum value(s) of this curve. Explain how you know you
are correct.
\end{prob}

\begin{prob}
Consider the curve $f(x) = 3(x-5)^2 +7$. Find the $x$ and $y$ values for
the maximum/minimum value(s) of this curve. Explain how you know you
are correct.
\end{prob}



\begin{prob}
Consider the curve $f(x) = -2(x+3)^2 + 7$. Find the $x$ and $y$ values for
the maximum/minimum value(s) of this curve. Explain how you know you
are correct.
\end{prob}

\begin{prob}
What type of curve is drawn by $f(x) = a(x-h)^2+k$? Find the $x$ and
$y$ values for the maximum/minimum value(s) of this curve. Explain how
you know you are correct.
\end{prob}

\begin{teachingnote}
This is the vertex form of a parabola.
\end{teachingnote}


\begin{prob}
Consider the parabola $f(x) = x^2 + 4x + 2$. Complete the square to
put this expression in the form above and identify the maximum/minimum
value(s) of this curve.
\end{prob}


\begin{prob}
Consider the parabola $f(x) = 2x^2 - 8x + 6$. Complete the square to
put this expression in the form above and identify the maximum/minimum
value(s) of this curve.
\end{prob}

\begin{prob}
Consider the parabola $f(x) = 3x^2 + 7x - 1$. Complete the square to
put this expression in the form above and identify the maximum/minimum
value(s) of this curve.
\end{prob}


\begin{prob}
Given a parabola $f(x) = ax^2 + bx +c$. Complete the square to put
this expression in the form above and identify the maximum/minimum
value(s) of this curve.
\end{prob}


\begin{prob}
Could you find the same formula found in the previous question by
appealing to the symmetry of the roots?
\end{prob}





