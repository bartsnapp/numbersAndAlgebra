\newpage
\section{The Triathlete}\label{A:Triathlete}

\begin{prob} 
On Friday afternoon, just as Laine got off the bus, she realized that she had left her bicycle at school.  In order to have her bicycle at home for the weekend, she decided to run to school and then ride her bike back home.  If she averaged 6 mph running and 12 mph on her bike, what was her average speed for the round trip?  Explain your reasoning. 
\end{prob}
\begin{prob}
On Saturday, Laine completed a workout in which she split the time evenly between running and cycling.  If she again averaged 6 mph running and 12 mph on her bike, what was her average speed for the workout?  Explain your reasoning. 
\end{prob}
\begin{prob}
Why was her average speed on Saturday different from her average speed on Friday?  Can you reason, without computation, which average speed should be faster?  
\end{prob}
\begin{prob} On Sunday, Laine's workout included swimming.  Assuming that she can swim at an average speed of 2 mph, describe two running-cycling-swimming workouts, one similar to Friday's scenario (same distance) and a second similar to Saturday's (same time).  Compute the average speed for each and explain your reasoning.  
\end{prob}
\begin{prob}
Which of the workout scenarios (same distance or same time) most closely resembles an actual triathlon?  Why do you think that is the case?
\end{prob}
\begin{prob}
After two months of intense training, Laine is able to average $s$ mph swimming, $r$ mph running, and $c$ mph cycling.  Again describe two running-cycling-swimming workouts, one similar to each of the two original scenarios, and compute her average speeds.     
\end{prob}
