\newpage
\section{Integer Multiplication}\label{A:integerMultiplication}



In this activity, we explore various models and strategies for 
making sense of multiplication of integers.  

\subsection*{Continuing patterns}
\begin{prob}
\begin{enumerate}
\item Continue the following patterns, and explain why it makes sense to continue them in that way.    

\begin{minipage}{0.45\textwidth}
\begin{align*}
4\times 3 &= 12 \\
4\times 2 &= \\
4\times 1 &= \\
4\times 0 &= \\
4\times (-1) &= \\
4\times (-2) &= \\
4\times (-3) &= \\
\end{align*}
\end{minipage}
\begin{minipage}{0.45\textwidth}
\begin{align*}
3\times 6 &= 18 \\
2\times 6 &= \\
1\times 6 &= \\
0\times 6 &= \\
(-1)\times 6 &= \\
(-2)\times 6 &= \\
(-3)\times 6 &= \\
\end{align*}
\end{minipage}
\begin{minipage}{0.45\textwidth}
\begin{align*}
(-7)\times 3 &= -21 \\
(-7)\times 2 &= \\
(-7)\times 1 &= \\
(-7)\times 0 &= \\
(-7)\times (-1) &= \\
(-7)\times (-2) &= \\
(-7)\times (-3) &= \\
\end{align*}
\end{minipage}

\item What rule of multiplication might a student infer from the first pattern? 
\item What rule of multiplication might a student infer from the second pattern?
\item What rule of multiplication might a student infer from the third pattern?
\end{enumerate}
\end{prob}

\subsection*{Using properties of operations}

\begin{prob}
Suppose we \emph{do not know} how to multiply negative numbers but we do know that $4\times 6=24$. We will use this fact and the properties of operations to reason about products involving negative numbers.  
\begin{enumerate}
\item What do we know about $A$ and $B$ if $A+B=0$?  
\item Use the distributive property to show that the expression $4\times 6 + 4\times(-6)$ is equal to $0$.  
Then use that fact to reason about what $4\times(-6)$ should be.  
\item Use the distributive property to show that the expression $4\times (-6) + (-4)\times (-6)$ is equal to $0$.  
Then use that fact to reason about what $(-4)\times(-6)$ should be.  
\end{enumerate}
\end{prob}

\subsection*{Walking on a number line}
\begin{teachingnote}
Again there are two decisions to make: (1) distinguishing positive and negative for the multiplicand; (2) distinguishing positive and negative for the multiplier.
\end{teachingnote}
\begin{prob} 
Matt is a member of the Ohio State University
  Marching Band. Being rather capable, Matt can take $x$ steps of size
  $y$ inches for all integer values of $x$ and $y$.  If $x$ is
  positive it means \textit{face North and take $x$ steps.} If $x$ is
  negative it means \textit{face South and take $|x|$ steps.} If $y$
  is positive it means your step is a \textit{forward step of $y$
    inches.} If $y$ is negative it means your step \textit{is a
    backward step of $|y|$ inches.}
\begin{enumerate}
\item Discuss what the expressions $x \cdot y$ means in this
  context. In particular, what happens if $x = 1$? What if $y=1$?
\item If $x$ and $y$ are both positive, how does this fit with the ``repeated addition'' model of multiplication?    
\item Using the context above and specific numbers, 
demonstrate the general rule:
\[
\text{negative}\cdot \text{positive} = \text{negative}
\]
Clearly explain how your problem shows this.
\item Using the context above and specific numbers, 
demonstrate the general rule:
\[
\text{positive}\cdot \text{negative} = \text{negative}
\]
Clearly explain how your problem shows this.\item Using the context above and specific numbers, 
demonstrate the general rule:
\[
\text{negative}\cdot \text{negative} = \text{positive}
\]
Clearly explain how your problem shows this.
\end{enumerate}
\end{prob}

