\newpage
\section{Prome Factorization}\label{A:Prome}

%The point of this activity is to see a number system in which unique factorization fails.  It turns out that Euclid’s Lemma also fails in this system, so it would help to begin the class with the following questions:  
%
%If 7|ab (where a and b integers), does it follow that 7 must divide either a or b? 
%If 14|ab (where a and b integers), does it follow that 14 must divide either a or b? 
%
%Through discussion, students should decide that the answers are “yes” and “no,” respectively, and the reason is that 7 is prime but 14 is not.  Some students should realize that, in the second case, the factors of 2 and 7 (of 14) might be “split” between a and b.  This realization can be summarized in the following statement: 
%
%Suppose a and b are integers and p is prime.  If p|ab, then p|a or p|b.  
%
%This fact is called Euclid’s Lemma, although the students will not be responsible for its name.  We accept it without proof.  
%
%The purpose of questions 1-4 is to see that this number system is very much like the integers:  You can always add, subtract, and multiply, but you cannot necessarily divide.  Students can use their reasoning about integers to explain these facts about the system 3Z.  
%
%The purpose of questions 5-7 is to notice that both Euclid’s Lemma and Unique Factorization fail in this number system.  Some examples:  
%
%36 = 3x13 = 6x6
%72 = 3x24=6x12

\fixnote{Incorporate \LaTeX comments into a teaching note.}  

\begin{teachingnote}
In the course, we first assume Euclid's Lemma and use it to prove the Fundamental Theorem of Arithmetic (FTA).  Here both Euclid's Lemma and the FTA fail.
\end{teachingnote}

Let's consider a crazy set of numbers---all multiples of $3$. Let's
use the symbol $3\Z$ to denote the set consisting of all multiples of
$3$. As a gesture of friendship, I have written down the first $100$
nonnegative integers in $3\Z$:

\[
\begin{array}{cccccccccc}
0   & 3   & 6   & 9   & 12  & 15  & 18  & 21  & 24  & 27  \\
\\
30  & 33  & 36  & 39  & 42  & 45  & 48  & 51  & 54  & 57  \\
\\
60  & 63  & 66  & 69  & 72  & 75  & 78  & 81  & 84  & 87  \\
\\
90  & 93  & 96  & 99  & 102 & 105 & 108 & 111 & 114 & 117 \\
\\
120 & 123 & 126 & 129 & 132 & 135 & 138 & 141 & 144 & 147 \\
\\
150 & 153 & 156 & 159 & 162 & 165 & 168 & 171 & 174 & 177 \\
\\
180 & 183 & 186 & 189 & 192 & 195 & 198 & 201 & 204 & 207 \\
\\
210 & 213 & 216 & 219 & 222 & 225 & 228 & 231 & 234 & 237 \\
\\
240 & 243 & 246 & 249 & 252 & 255 & 258 & 261 & 264 & 267 \\
\\
270 & 273 & 276 & 279 & 282 & 285 & 288 & 291 & 294 & 297
\end{array}
\]



\begin{prob}
Given any two integers in $3\Z$, will their sum be in $3\Z$? Explain
your reasoning.
\end{prob}

\begin{prob}
Given any two integers in $3\Z$, will their difference be in $3\Z$?
Explain your reasoning.
\end{prob}

\begin{prob}
Given any two integers in $3\Z$, will their product be in $3\Z$?
Explain your reasoning.
\end{prob}

\begin{prob}
Given any two integers in $3\Z$, will their quotient be in $3\Z$?
Explain your reasoning.
\end{prob}

\begin{definition}
Call a positive integer \textbf{prome} in $3\Z$ if it cannot be
expressed as the product of two integers \textit{both} in $3\Z$.
\end{definition}

As an example, I tell you that $6$ is prome number in $3\Z$. You may
object because $6 = 2\cdot 3$, but remember---$2$ is not in $3\Z$!


\begin{prob}
List some of the prome numbers less than $297$.  Hint:  What numbers in $3\Z$ \emph{can} be expressed as a product of two integers \emph{both} in $3\Z$?  
\end{prob}

\begin{prob}
Can you give some sort of algebraic characterization of prome numbers
in $3\Z$? 
\end{prob}

\begin{prob}
Can you find numbers that factor completely into prome numbers in
\textit{two} different ways? How many can you find?
\end{prob}




