\newpage
\section{Shelby and Scotty}\label{A:SS}

Shelby and Scotty want to express the number $27$ in base
$4$. However, they used very different methods to do this. Let's check
them out.

\fixnote{Perhaps swap the order.}




\begin{prob} Consider Shelby's work:
\[
4\,\begin{tabular}[b]{@{}r@{} r}
$6$ &\, R\,$\boldsymbol{3}$\\ \cline{1-1}
\Big)\begin{tabular}[t]{@{}l@{}}
$27$ 
\end{tabular}
\end{tabular}
\qquad
4\,\begin{tabular}[b]{@{}r@{} r}
$1$ &\, R\,$\boldsymbol{2}$\\ \cline{1-1}
\Big)\begin{tabular}[t]{@{}l@{}}
$6$ 
\end{tabular}
\end{tabular}
\qquad
4\,\begin{tabular}[b]{@{}r@{} r}
$0$ &\, R\,$\boldsymbol{1}$\\ \cline{1-1}
\Big)\begin{tabular}[t]{@{}l@{}}
$1$ 
\end{tabular}
\end{tabular} \qquad \Rightarrow \qquad \fbox{$123$}
\]
\begin{enumerate}
\item Describe how to perform this algorithm.
\item Provide an additional relevant and revealing example
  demonstrating that you understand the algorithm.
\end{enumerate}
\end{prob}




\begin{prob} Consider Scotty's work:
\[
4^3\,\begin{tabular}[b]{@{}r@{} r}
$0$ &\, R\,$27$\\ \cline{1-1}
\Big)\begin{tabular}[t]{@{}l@{}}
$27$ 
\end{tabular}
\end{tabular}
\qquad
4^2\,\begin{tabular}[b]{@{}r@{} r}
$\boldsymbol{1}$ &\, R\,$11$\\ \cline{1-1}
\Big)\begin{tabular}[t]{@{}l@{}}
$27$ 
\end{tabular}
\end{tabular}
\qquad
4\,\begin{tabular}[b]{@{}r@{} r}
$\boldsymbol{2}$ &\, R\,$\boldsymbol{3}$\\ \cline{1-1}
\Big)\begin{tabular}[t]{@{}l@{}}
$11$ 
\end{tabular}
\end{tabular} \qquad \Rightarrow \qquad \fbox{$123$}
\]
\begin{enumerate}
\item Describe how to perform this algorithm.
\item Provide an additional relevant and revealing example
  demonstrating that you understand the algorithm.
\end{enumerate}
\end{prob}

\begin{prob} 
Create an illustration (or series of illustrations) based on the $27$
marks below that models Shelby's method for changing bases.
\[
|\;\;|\;\;|\;\;|\;\;|\;\;|\;\;|\;\;|\;\;|\;\;|\;\;|\;\;|\;\;|\;\;|\;\;|\;\;|\;\;|\;\;|\;\;|\;\;|\;\;|\;\;|\;\;|\;\;|\;\;|\;\;|\;\;|
\]
Further, explain why Shelby's method works. 
\end{prob}

\begin{prob} 
Create an illustration (or series of illustrations) based on the $27$
marks below that models Scotty's method for changing bases.
\[
|\;\;|\;\;|\;\;|\;\;|\;\;|\;\;|\;\;|\;\;|\;\;|\;\;|\;\;|\;\;|\;\;|\;\;|\;\;|\;\;|\;\;|\;\;|\;\;|\;\;|\;\;|\;\;|\;\;|\;\;|\;\;|\;\;|
\]
Further, explain why Scotty's method works. 
\end{prob}

\begin{prob}
Use both methods to write 1644 (base ten) in base seven.
\end{prob}

\begin{prob}
Now lets try to be more efficient.  
\begin{enumerate}
\item Convert 8630 (base ten) to base thirteen.  Use A for ten, B for eleven, and C for twelve.  
\item Quickly convert 2102 (base three) to base nine.
\item Without using base ten, convert 341 (base six) to base four.  
\item Without using base ten, convert 341 (base six) to base eleven.
\end{enumerate}
\end{prob}

