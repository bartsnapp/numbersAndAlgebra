\newpage
\section{Sketching Roots}\label{A:sketchRoots}

In this activity we seek to better understand the connection between
roots and the plots of polynomials.

\begin{prob}
Sketch the plot of a quadratic polynomial with real coefficients that has:
\begin{enumerate}
\item Two real roots.
\item One repeated real root.
\item No real roots.
\end{enumerate}
In each case, give an example of such a polynomial.
\end{prob}

\begin{prob}
Can you have a quadratic polynomial with exactly one real root and
$1$ complex root?  Explain why or why not.
\end{prob}


\begin{prob}
Sketch the plot of a cubic polynomial with real coefficients that has:
\begin{enumerate}
\item Three distinct real roots.
\item One real root and two complex roots.
\end{enumerate}
In each case, give an example of such a polynomial.
\end{prob}

\begin{prob}
Can you have a cubic polynomial with no real roots?  Explain why or
why not. What about two distinct real roots and one complex root?
\end{prob}


\begin{prob}
For polynomials with real coefficients of degree $1$ to $5$, classify
exactly which types of roots can be found. For example, in our work
above, we classified polynomials of degree $2$ and $3$.
\end{prob}
