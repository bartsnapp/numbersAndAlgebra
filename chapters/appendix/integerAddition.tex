\newpage
\section{Integer Addition and Subtraction}\label{A:integerAddition}

In this activity, we explore various models and strategies for 
making sense of addition and subtraction of integers.  

\subsection*{Useful language}
Addition and subtraction problems arise in situations where we add to, take from, put together, 
take apart, or compare quantities.  

Recall that addition and subtraction facts are related.  For example, if we know that $8+5 = 13$, 
then we also know three related facts:  $5+8=13$, $13-8=5$, and $13-5=8$.  In school mathematics, 
these are often called \emph{fact families}.  

\begin{prob}
What are integers?  Describe some situations in which both positive and negative integers arise.  Use the word ``opposite'' in your descriptions.  
\end{prob}

\subsection*{Red and black chips}
\begin{prob}
In a red-and-black-chip model of the integers, red and black chips each count for $1$, but they are opposites, so that they cancel each other out.  Using language from accounting, suppose black chips are assets and red chips are debts.  We add by putting chips together.  Use red and black chips (or draw the letters $R$ and $B$) to model the following computations.
\begin{enumerate}
\item $(-5)+(-3)$
\item $6+(-4)$.
\item $(-7)+9$
\item $2+(-5)$
\end{enumerate}
\end{prob}

\begin{prob}
In the previous problem, you saw different combinations of red and black chips that had the same numerical value.  
\begin{enumerate}
\item How many ways are there to represent $-3$?  Draw two different representations. 
\item Use the phrase ``zero pairs'' to describe how your two representations are related.  
\end{enumerate}
\end{prob}

\begin{prob}
To subtract in the red-and-black-chip model, we can ``take away'' chips, as you might expect.  When we don't have enough
chips of a particular color, we can always add ``zero pairs.''  Use this idea to model the following subtraction problems: 
\begin{enumerate}
\item $6-8$
\item $4-(-3)$
\item $(-6)-5$
\item $(-3)-(-7)$
\end{enumerate}
\end{prob}

\subsection*{Subtraction as missing addend}
\begin{prob}
To evaluate a subtraction expression, we can solve a related addition equation.  For example, $11-7$ is the solution to $7+\rule[-2pt]{12pt}{.5pt} =11$.  Use this idea to evaluate the subtraction expressions in the previous problem.  
\end{prob}

\subsection*{Subtraction as difference on the number line}
\begin{prob}
Use a number line to reason about $b-a$ by asking how to get from $a$ to $b$:  How far?  And in which direction?   For example, to evaluate $11-7$, we can ask how to get from $7$ to $11$.  We travel $4$ units to the right.  Use this idea to evaluate the subtraction expressions in the previous problems.
\end{prob}

\begin{prob}
How is subtraction different from negation?  
\end{prob}

\begin{prob}
Use what you have learned to explain why $a-(-b)=a+b$.  
\end{prob}

\subsection*{Other Models}  
Use the following models for addition and subtraction of integers.  Each model requires two decisions:  (1) how positive and negative integers are `opposite' in the situation, and (2) how addition and subtraction are `opposite' in a different way.  
\begin{itemize}
\item A postal carrier who brings checks and bills---and who also takes them away.  
\item Walking on an North-South number line, facing either North or South, and walking either forward or backward.  
\end{itemize}

