\newpage
\section{Close, but No Cigar}

Sometimes, we are not able to come up with formulas for our function
models.  In fact, sometimes all that we have are a few data points $(x,
y)$.  Obviously, we'd like to know more exact values of our function,
but the cupboard is often bare---we have to make lemonade out of
lemons.  Thus, we try to ``fill in'' points in between the given points
in order for our model to have some predictive quality to it.

You already saw this in Statistics 145 when you talked about the
``best-fit'' line.  There actually could be several definitions of what
is meant by this, but typically we use the ``least-squares'' line.
Although we usually see the formula for the slope and intercept of
this line either just quoted or derived with multi-variable calculus
in textbooks (something probably not done with the limited time
allotted for Math 111), you can actually find it for yourself using
algebraic ideas from school (although it also would be nice to know
why those algebraic ideas work!):

\begin{prob}
Find the ``least-squares'' line for the data points $(2, 3)$, $(4,
5)$, and $(6, 11)$ (note that these don't already form a line- graph
or algebraically see or just use common sense): That is, find $a$ and
$b$ so that the sum of the squares of the vertical distances of the
above data points from the line $y=ax+b$ is as small as possible
(Unpack what these words really say---a graph is probably best!  Then
solve algebraically using what you know from Algebra II or so).
\end{prob}

\begin{prob}
Of course, contrary to the ``linear disease'' that engulfs our
country, not every variable grows at the same rate all the time (i.e.,
we're not always on cruise control or, mathematically, not every
relationship is linear!)!  If the data you're given appear to be
non-linear, we try to fit a predictive non-linear curve to such a
graph (e.g., if the graph appears to be exponential in nature, we try
to fit a ``best-fit exponential function'' $y=ke^{ax}$ instead of a
``best-fit'' line $y=ax+b$).  Now how can you tell if a graph appears
to be exponential or more like a polynomial?  For example, look at the
data below:
\[
\begin{array}{|c|c|c|c|c|}\hline
x & 0 & 1 & 2 & 3 \\ \hline
y & 8.1 & 22.1 & 60.1 & 165 \\ \hline
\end{array}
\]

A group of students graph the data.  Susie says, ``This sure looks
like a parabola!''  But Fred says, ``No it doesn't, I just got done
studying exponentials and it looks like an exponential!''  Susie says
in reply, ``What?! I sat in those boring classes with you---I know
about exponentials, but I also know my parabolas---and this is a
parabola.''  Then Julie chimes in, ``Both of you dummies are wrong,
it's a quartic ($x^4$)''.  Then a whole slew of ``Says who? Says me!''
ensues.  Before this turns into a brawl and people get hurt, how can
we sanely tell who is right, if any of them? They can't all be right!


Hint: Think about the most famous graph of all, the one you know most
about.  And see if you can somehow convert the above data to get that
type of graph by graphing the relationship between variables related
to $x$ and $y$ That is, if it was truly best modeled by $y=ke^{ax}$,
then we'd know (why?) $\ln(y) = ax \ln(k)$ (try graphing the
$x$-coordinates on the horizontal axis and the natural log of the $y$
coordinates on the vertical axis. What should that graph look like if
it is best modeled by an exponential?).  If it were best modeled by $y
= kx^a$ (i.e., like a polynomial), then we'd know (why?) $\ln(y) =
a\ln(x) + \ln(k)$ (what would you graph on each axis here?).  Graph
each of these with the given data and see what happens (and find best
values for either $k$ or $a$ for the model you conjecture is best.  Note:
We could ask you to find the ``best-fit'' line for the ``linearized'' data
like in \#1, but we'll give you a break and just ask for a good eyeball
estimate here!)).

Now do the same with this data:
\[
\begin{array}{|c|c|c|c|c|}\hline
x & 1 & 2 & 3 & 4 \\ \hline
y & 8.3 & 443.6 & 24420.8 & 1364278.6 \\ \hline
\end{array}
\]

Now do the same with this data:
\[
\begin{array}{|c|c|c|c|c|c|}\hline
x & 1 & 2 & 3 & 4 & 5 \\ \hline
y & 7 & 62 & 220 & 506 & 1012 \\ \hline 
\end{array}
\]
\end{prob}

\begin{prob}
Although not all data are modeled by polynomials, because of their
easy function evaluations (place value, all K-8 arithmetic!) and
smoothness of graphs, polynomials are often a universal go-to model
for data (we try to fit a polynomial that contains all the data
points).  As was noted in Polly Nomial \#3, there are a couple of ways
to do this, at least one of which you'll do in a future activity.

Note: sometimes, finding ``best-fit'' curves can be done by a graphing
calculator (On the TI-83 and its kin, go to STATS, enter your
(2-variable) data into EDIT, and go to CALC (where you'll find LinReg,
CubicReg (cubic polynomial), ExpReg , etc.)  Of course, the
graphing calculator will blindly give you the ``best-fit'' curve of
all ilk's.  It still takes, as illustrated above, a human mind to
figure out which, if any, best fit the given data.for purposes of
prediction.
\end{prob}
