\newpage
\activity\section{The Harmonic Triangle}\index{harmonic triangle} 

Leibniz, one of the inventors of calculus, dreamed up this beast while
trying to solve a problem posed to him by Huygens:
\[
\begin{tabular}{@{\,}c@{\,}c@{\,}c@{\,}c@{\,}c@{\,}c@{\,}c@{\,}c@{\,}c@{\,}}
& & & & $\frac{1}{1}$ & & & & \\
& & & $\frac{1}{2}$ & & $\frac{1}{2}$ & & & \\
& & $\frac{1}{3}$ & & $\frac{1}{6}$ & & $\frac{1}{3}$ & & \\
& $\frac{1}{4}$ & & $\frac{1}{12}$ & & $\frac{1}{12}$ & & $\frac{1}{4}$ &\\
$\frac{1}{5}$ & & $\frac{1}{20}$ & & $\frac{1}{30}$ & & $\frac{1}{20}$ & & $\frac{1}{5}$
\end{tabular}
\]

\begin{prob} 
What relationships can you find between the entries of the triangle as
we move from row to row?
\end{prob}


\begin{prob} What are the next two rows? Clearly articulate how to produce more rows of the Harmonic Triangle.
\end{prob}

\begin{prob} Explain how the following expression
\[
\frac{1}{k\binom{n}{k}}
\]
corresponds to entries of the Harmonic Triangle. Feel free to draw
diagrams and give examples.
\end{prob}

\begin{prob}  
Explain how the Harmonic Triangle is formed. In your explanation, use
the notation
\[
\frac{1}{k\binom{n}{k}}
\]
If you were so inclined to do so, could you state a single equation
that basically encapsulates your explanation above?
\end{prob}

\begin{prob} 
Can you explain why the numerators of the fractions in the Harmonic
Triangle must always be $1$?
\end{prob}


\begin{prob} Explain how to use the Harmonic Triangle to go from:
\[
\frac{1}{2} + \frac{1}{6} + \frac{1}{12} + \frac{1}{20}+\cdots
\]
to
\[
\left(1 - \frac{1}{2}\right) + \left(\frac{1}{2}-\frac{1}{3}\right) + 
\left(\frac{1}{3}- \frac{1}{4}\right) + \left(\frac{1}{4}-\frac{1}{5}\right) + \cdots 
\]
Conclude by explaining why Leibniz said:
\[
\frac{1}{2} + \frac{1}{6} + \frac{1}{12} + \frac{1}{20}+\cdots = 1
\]
\end{prob}

\begin{prob} Explain how to use the Harmonic Triangle to go from:
\[
\frac{1}{3} + \frac{1}{12} + \frac{1}{30} + \frac{1}{60}+\cdots 
\]
to
\[
\left(\frac{1}{2} - \frac{1}{6}\right) + \left(\frac{1}{6}-\frac{1}{12}\right) + 
\left(\frac{1}{12}- \frac{1}{20}\right) + \left(\frac{1}{20}-\frac{1}{30}\right) + \cdots 
\]
Conclude by explaining why Leibniz said:
\[
\frac{1}{3} + \frac{1}{12} + \frac{1}{30} + \frac{1}{60}+\cdots = \frac{1}{2}
\]
\end{prob}

\begin{prob} 
Can you generalize the results above? Can you give a list of infinite
sums and conjecture what they will converge to?
\end{prob}





%%%%%%%%%%%%%%%%%%%%%%%%%%%%%%%%%%%%%%%%%%%%
%%%%%%%%%%%%%%%%%%%%%%%%%%%%%%%%%%%%%%%%%%%%
%% REF - Cut the knot
%% http://www.cut-the-knot.org/Curriculum/Combinatorics/LeibnitzTriangle.shtml
%% The Origins of the Infinitesimal Calculus. Margaret E. Baron. Dover 1969  
%%
