\newpage
\section{Undoing---Using Inverse Functions}



\begin{prob}
Formulas are fine, but what do you do to solve something like this:
\[
5 = e^x \qquad \text{}
\]
\end{prob}





and To Infinity and Beyond!)

What about equations that can't be put (or easily) put into a nice
form?  They might have solutions too, but if we only rely on the
``undoing'' means, we may never find them (and thus, all most students
are given in school are those ``rigged up'' equations- yes- important
to know how to solve, but they're not the only ones around)!  For
example, if you involve two or more types of functions in the same
equation (say exponential, log, and polynomial- e.g., $5\cdot
2^{x-4}-x^13 + \ln(x) + 102.8=0$), you now have several different
kinds of ``undoings'' to contend with.

Over the centuries, methods have been developed to deal with the
``ugly'' cases mentioned in the paragraph above.  Usually, we only end
up with an approximate answer, but we can often dictate how close the
answer can be to the actual (by analyzing the algorithms and figuring
out how many steps would be necessary to get as close as we need to
be).

\paragraph{Graphing Calculator} 
With the advent of powerful technology, we can often sidestep many of
these mathematical issues with even a simple hand-held graphing
calculator (which probably has a ``solve'' feature that will (usually)
give you a solution closest to a value you give it (if a solution
exists)).  You can also do a little more yourself by using the
graphing feature and (best done by bringing everything over = 0 and
enter the function on the other side and find its x-intercepts by
using the ``zoom'' feature).  Side question, though: How is the
calculator doing these things?  How was it programmed?  What
mathematical methods is it using?

However, you can also do a few other techniques that require a bit
more work.  For example:

\paragraph{Judicious Guess and Check}  
Don't underestimate this technique, which is the first technique that
we really should start with in early childhood (e.g., ``Fill in the
brackets: $15-\langle\, \rangle=6$''.  Before knowing any ``undoing''
techniques, we guess and check judiciously until we see that $9$ is what
makes the statement true- perhaps using objects and/or couching it in
a story problem.).  It can also lead to an expression(s) to model the
story problem if you can't already find one (perhaps it helped you in
the \textit{Dreaded Story Problem} to do just that!).  

For solving the equations that are wacky like the one above that
involved a lot of function types, we basically do the same as the
early childhood student (with a little more background) and can use
the calculator to do the grungy calculating for us.  Enter and guess
our first solution to be, say, 4 and see if that gives us zero when we
plug it in.  Unfortunately, $f(4) = -67108754.81$, which apparently is
not zero!  But we guess again and plug in, say, 1 and we get $f(1) =
102.425$.  I claim that we can guarantee that there is a solution
between $x = -1$ and $x = 4$ (why?).  Now hone in on that solution (or
maybe even find others in that interval).  We'll do the ``bisection''
technique here, which means we'll test the midpoint between $1$ and $4$
(and all midpoints of subsequent intervals): 2.5: $f(2.5) \approx -150000$. So, we have an answer between 1 and 2.5 (why?).  Try $f(1.75)=
-1339.348$.  Try 1.375 (midpoint between 1 and 1.75); $f(1.375)=41.132$,
which is greater than zero, so our answer must be between 1.375 and
1.75.  Try that midpoint: 1.5625: $f(1.5625)=-226.703<0$.  So the answer
must be between 1.375 and 1.5625: try 1.46875: $f(1.46875)=-43.9713<0$.
So we know our answer must be between 1.46875 and 1.375.  And so on
until we get the accuracy we need.  There are several variations on
this technique (in terms of how to find your next guess.  Here we
always used the midpoint).

\paragraph{Fixed Point Iteration} 
(You might look at the activity ``Spreadsheet Mania'' before embarking
on this method).  Often, when teaching algebra to students who only
know how to solve linear equations, I will find students who make an
initial stab at solving, say, a quadratic equation for $x$, in the
following manner:
\begin{align*}
x^2 + 2x -8 &= 0 & \\
2x &= 8 - x^2 & \\
x &= 4- .5 x^2 & \text{(divided by $2$)}
\end{align*}
Or, they will do something like this:
\begin{align*}
x^2 + 2x -8 &= 0 \\
x(x+2) &= 8 \\
x = \frac{8}{x+2}
\end{align*}
Of course, this seems pointless, as we don't get an equation of the
tried and true form ``x = a number''.  Instead, we get 
\[
x= \text{another expression in terms of $x$} 
\]
or ``x = g(x).'' What should we do with such students?  Suspend or
expel them???! Actually, the students are not wrong (other than not
finding a numerical answer).  These are re-writings (``interior
decorating'') of the original equation.  But what good are they?

It turns out that, under certain conditions, we can take advantage of
this form to get a solution (but not necessarily all solutions) by
using this form $g(x)=x$ and plugging in an initial value into $g(x)$
and plugging the result back into g(x), and so on.  Under certain
conditions, the values will approach a solution (i.e., we keep getting
roughly the same value over and over, meaning the $g(x)$ we're getting
is the same as the $x$ we'll plug in next, thus satisfying the
original equation).  And it's a nice method for when $x$ can be solved
for using traditional algebraic techniques (``undoing''), but just not
for a number.
      
You can use your calculator or (much quicker yet) the spreadsheet to
do this (How is this related to recursion??).
      
In our example, we already know the solutions to the equation from previous schoolwork.  That is, $x = 2$ and $x = - 4$.
\begin{enumerate}
\item Use $x=\frac{8}{x+2}$ .  Start with $x = 4$ and plug it into the right side (the initial value is often called the ``seed'').  We get $8/6= 1.\bar{3}$.  Now plug $1.\bar{3}$ into the same expression and get $2.4$.  Then plug in $2.4$ into it, etc., etc.  Keep doing this about 10 times (10 iterations) and you'll notice that the resulting values bounce before and after 2, one of our known solutions.  In fact, you'll find that no matter what your seed is (even, say, -4000), you'll relatively quickly be converging to 2.  However, no seed will get you close to the other solution (-4).  (How can this be done on a spreadsheet?: See the later activity ``Spreadsheet Mania'' for a good shortcut method to avoid the drudgery you've experienced)
\item Try the other ``solution'' we came up with in the intro:  $g(x) = 4-.5x^2$  Try it, no matter what you put in for your seed or the number of iterations, it won't get close to either solution (in fact, it sometimes will ``blow up'').  Thus, some forms are better than others for this method.
\item Say we wanted to solve $x^3 – 3x = 0$.  (We can do this
conventionally by factoring out $x$ and getting $0$, and +/- square root
of 3 (about 1.732)).  We can also ``solve for x'' and get $x = x^3 –
2x$.  Using $x^3-2x$ as our $g(x)$, use 1 as your seed and what happens?
Will any seed or \# of iterations get us reasonably close to the known
roots?  Try an alternative: $x = (1/3)x^3$. (Try seeds = 1.732, 1.735,
1.738.  What happened?  Try seed=1.7.  What happened?)  

\item Try another quadratic equation (known solutions again): $x^2+9x+19 = 0$.  Solve for $x = -19/(x+9)$ and use several seeds.  Do you get close to a solution?
Which one?  

\item Now solve an equation we have no idea about a solution
for: $x^2 – \cos(x) = 0.$  Solving for $x$ gives $x = \sqrt{\cos(x)}$.  Can you
conjecture a solution?  Check that it works in the original equation
on a calculator. 
\item When does this ``fixed point'' technique work?
Put the graphs of our $g(x)$'s with the graph $y=x$ (Why would this make
sense to do?).  Because we're not as familiar with some of the $g(x)$'s,
let's use g(x) = linear graphs first.  Use the spreadsheet as well:
Use $g(x) = 2x +3, -2x + 3, .7x + 3, 1.1x +3, .92x + 3$., etc. with
several seeds.  You'll have to fill down quite a ways (say 1000) to
see what's happening.  Sketch the graph of each with y=x and see what
happens graphically.  Make a conjecture as to when fixed point
iteration works or not.  

\item This idea of ``fixed point'' is
generalized to more powerful equation solving algorithms, including
Newton's Method.  It is generalized even more to solving other kinds
of equations, such as differential equations.
\end{enumerate}
\paragraph{Calculus-Based Techniques} For you calculus fans out there, another famous equation-solving method is ``Newton's Method''.  It's reasonably simple to understand- just need to see a picture involving tangent lines to see the reasoning. We might visit this in Math 111.
