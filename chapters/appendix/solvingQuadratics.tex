\newpage
\section{Solving Quadratics}\label{A:solvingQuadratics}
Here we explore various methods for solving quadratic equations in one variable.  \textbf{Please read all instructions carefully.}

\begin{prob}
Is $\sqrt{4}=\pm 2$?  Explain. 
\end{prob}

\vfill

\begin{teachingnote}
Both $2$ and $-2$ are ``square roots'' of $4$ because $2^2=4$ and $(-2)^2=4$, and both of them are solutions to the equation $x^2=4$.  The question is whether the radical symbol refers to both of them, either of them (you choose?), or a specific one of them.  
\end{teachingnote}

\begin{prob}
Suppose that $\sqrt{4}=\pm 2$?  Then evaluate $\sqrt{4}+\sqrt{9}$.  
\end{prob}

\begin{teachingnote}
$$\sqrt{4}+\sqrt{9}=\pm2+\pm3=5, -1, 1, or -5$$
\end{teachingnote}

\vfill

\begin{prob}
What does your calculator say about $\sqrt{4}+\sqrt{9}$?  
\end{prob}

\vfill 

\begin{teachingnote}
Now emphasize the conventional meaning of the radical symbol:  For $a>0$ then $\sqrt{a}$ means the positive square root of $a$.  
\end{teachingnote}



\begin{prob}
In the following problems, you \textbf{may not use the quadratic formula}.  But just for the record, write down the quadratic formula.  
\end{prob}
\begin{teachingnote}
Many students will write only $\frac{-b\pm\sqrt{b^2-4ac}}{2a}$, but we want them to write the following:  
$$\text{If }ax^2+bx+c=0\text{ and }a\ne 0\text{, then }x=\frac{-b\pm\sqrt{b^2-4ac}}{2a}\text{.}$$
Note that if the radical symbol were to refer to both a positive and negative square root, then there would be no reason to write $\pm$ outside the radical symbol.  
\end{teachingnote}
\vspace{0.8in}

\begin{prob}
In the following list of equations, solve those that are \textbf{easy} to solve.  
\begin{enumerate}
\item $(x-3)(x+2)=0$
\item $(x-3)(x+2)=1$
\item $(2x-5)(3x+1)=0$
\item $(x-a)(x-b)=0$
\item $(x-1)(x-3)(x+2)(2x-3)=0$
\end{enumerate}
\end{prob}

\begin{prob}
Regarding the previous problem, state the property of numbers that made all but one of the equations easy to solve.  
\end{prob}

\begin{teachingnote}
Zero product property:  If $ab = 0$ then $a=0$ or $b=0$.  Note that this doesn't work when the right side is not 0.  
\end{teachingnote}

\vspace{0.3in}


%\begin{prob} For each part below, write a linear equation with the state solution.  
%\begin{enumerate}
%\item $x=\frac{2}{3}$
%\item $x=\frac{a}{b}$
%\end{enumerate}
%\end{prob}

\begin{prob}For each part below, write a quadratic equation with the stated solution(s) and no other solutions.  
\begin{enumerate}
\item $x=7$ or $x=-4$
\item $x=p$ or $x=q$
\item $x=3$
\item $x=\frac{1\pm\sqrt{5}}{2}$
 \end{enumerate}
\end{prob}

\begin{prob}
Find all solutions to $x^3-3x^2+x+1=0$.  Hint:  One solution is $x=1$.  
\end{prob}
