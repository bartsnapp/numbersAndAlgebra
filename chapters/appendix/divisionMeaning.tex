\newpage
\section{What Can Division Mean?}\label{A:dm}

Here are some problems involving division. Someone once told me that
most division problems could be broken into two types:
\begin{enumerate}
\item Those that are asking ``How many groups?''
\item Those that are asking ``How many in each group?''
\end{enumerate}
Let's put this claim to the test. For each of the problems below:
\begin{enumerate}
\item Numerically solve the problem. Should our answer be a decimal, or a quotient and reminder?
\item Draw a picture representing the situation and describe actions with objects a student could carry out to solve the problem.
\item Identify whether the problem is asking ``How many groups?'' or ``How many in each group?'' or something else entirely.
\end{enumerate}

\begin{prob}
There are a total of $35$ hard candies. If there are $5$ boxes with an
equal number of candies in each box---and all the candy is accounted
for, then how many candies are in each box? What if you had $39$
candies?
\end{prob}

\begin{prob}
There are a total of $28$ hard candies. If there are $4$ candies in
each box, how many boxes are there? What if you had $34$ candies?
\end{prob}


\begin{prob}
There is a total of 29 gallons of milk to be put in 6 containers.  If
each container holds the same amount of milk and all the milk is
accounted for, how much milk will each container hold?
\end{prob}

 
\begin{prob}
There is a total of 29 gallons of milk to be put in containers holding
6 gallons each.  If all the milk is used, how many containers were
used?
\end{prob}
 
\begin{prob}
If there were 29 kids and each van holds 5 kids, how many vans would
we need for the field trip?
\end{prob}

%\begin{prob}
%Paolo has a total of $48$ outfits (shirts and pants) he can wear. If
%he has $8$ shirts, how many pants does he have?
%\end{prob}



%\begin{prob}
%A chart has $72$ cells and $8$ rows. How many columns does it have?
%\end{prob}

%\begin{prob} 
%A rectangle has a length of $6$ inches and an area of $42$ square
%inches. What is its width?
%\end{prob}

%\begin{prob} 
%Can you think of a division problem that is fundamentally different
%from the problems above?
%\end{prob}

%\begin{prob} 
%In the context of the problems above, what might ``division by zero''
%mean?
%\end{prob}


