\newpage
\section{You Can Count on It!}\label{A:countOnIt}

\fixnote{Incorporate LaTeX comments into a teaching note.}

% The words permutation and combination tend to promote formulaic thinking rather than careful reasoning.  And students tend to ask “does order matter” in ways that often don’t help them with the reasoning. 
%
%What I think they need are: (1) the multiplication principle of counting (and this gives them permutations for free, without the name); and (2) strong explanatory thinking around “n choose k,” connecting them to both traffic lights and Pascal’s triangle.  
%
%  See the final exam review document from 2014.  

%Tomorrow Delanie will aim to complete Activity A45.  These are notes about a conversation we had earlier this evening.  
%
%The major point/purpose is the fundamental counting principal and the ability to use it thoughtfully (i.e., supported by thinking) rather than formulaically.  Although we will find formulas for (n choose k) to be useful, we need not emphasize formulas for, say, permutations.
%
%We found that the activity often jumps into the deep end.  So here are some ways we decided to soften it a bit. 
%
%First, it needs a simple introduction:  
%
%A.45.0.  The Diet-Lite restaurant offers 5 entrees and 3 side dishes.   Assuming one of each, how many different dinners can you order?
%
%The primary purpose to illustrate the multiplication principle for a situation where you can draw the whole tree diagram.  The secondary purpose is to illustrate the utility of solving an easy problem before solving a hard one.  The students might find it useful in some of the later problems (e.g., 45.6 and 45.7) to solve easier problems first.  
%
%Second, two of the problems should be omitted, at least the first time through:  
%
%A.45.4. Pizza toppings is best saved until later, after the formula for (n choose k) has been firmly established.  Then it can make very useful connections. 
%
%A.45.8.  The crazy dinner orders problem is too involved for an activity.  Much better as a homework problem.  


\begin{prob}
The Diet-Lite restaurant offers 5 entrees, 8 side dishes, 12 desserts,
and 6 kinds of drinks.  If you were going to select a dinner with one
entr\'ee, one side dish, one dessert, and one drink, how many different
dinners could you order?
\end{prob}

\begin{prob}
A standard Ohio license plate consists of two letters followed by two
digits followed by two letters.  How many different standard Ohio
license plates can be made if: 
\begin{enumerate}
\item There are no more restrictions on the
numbers or letters.
\item  There are no repeats of numbers or letters.
\end{enumerate}
\end{prob}

\begin{prob}
Seven separate coins are flipped.  How many different results are
possible (e.g., HTHHTHT is different than THHHTTH)?
\end{prob}


\begin{prob}
A pizza shop always puts cheese on their pizzas.  If the shop offers
$n$ additional toppings, how many different pizzas can be ordered
(Note: A plain cheese pizza is an option)?
\end{prob}

\begin{prob}
There are 10 students in the auto mechanics club.  Elections are
coming up and the members are holding nominations for President, Vice
President, Secreatary, and Treasurer.  If all members are eligible,
how many possible tickets are there?
\end{prob}

\begin{prob}
Same as the previous question, but now there are $n$ members of the club
and $k$ offices.
\end{prob}


\begin{prob}
Now the club (with $n$ members) is not electing officers anymore, but
instead deciding to send $k$ delegates to the state auto mechanics
club convention.  How many possible groups of delegates can be made?
\end{prob}

\begin{prob}
The Pig-Out restaurant offers 5 entrees, 8 side dishes, 12 desserts,
and 6 kinds of drinks.  If you were going to select a dinner with 3
entr\'ees, 4 side dishes, 7 desserts, and one drink, how many
different dinners could you order?
\end{prob}
