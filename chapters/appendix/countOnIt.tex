\newpage
\section{You Can Count on It!}\label{A:countOnIt}
\begin{teachingnote}
 The words permutation and combination tend to promote formulaic thinking rather than careful reasoning.  And students tend to ask ``Does order matter?'' in ways that don't help them with the reasoning. 

The major purposes of this activity: (1) the multiplication principle of counting, supported by a drawn or imagined tree diagram; and (2) strong explanatory thinking around ``$n$ choose $k$,'' connecting to both traffic lights and Pascal's triangle.  

Although students will find formulas for ``$n$ choose $k$'' to be useful, we need not emphasize formulas for permutations because the multiplication principle provides it for free.   (See also the final exam review document from 2014.)
\end{teachingnote}

\begin{prob}
The Diet-Lite restaurant offers 3 entr\'ees, 4 side dishes, 2 desserts
and 5 kinds of drinks.  
\begin{enumerate}
\item If you were going to select a dinner with one
entr\'ee and one side dish, how many different dinners could you order?  Explain your reasoning.  
\item If you were going to select a dinner with one
entr\'ee, one side dish, one dessert, and one drink, how many different
dinners could you order?
\end{enumerate}
\end{prob}

\begin{prob}
Suppose an Ohio license plate consists of two letters followed by two
digits followed by two letters.  How many different
license plates can be made if: 
\begin{enumerate}
\item There are no more restrictions on the
numbers or letters.
\item  There are no repeats of numbers or letters.
\end{enumerate}
\end{prob}

\begin{prob}
Naming officers and choosing a committee.
\begin{enumerate}
\item How many ways can a chairperson, secretary, and treasurer be named in a club of 10 people?  
\item How many ways can a committee of 3 people be chosen from this same club?
\item Explain using how the answer to (b) makes sense by beginning with the answer to (a) and then ``adjusting'' for overcounting.  
\item Generalize part (c) to explain a formula for the number of ways that a committee of $k$ people can be chosen from a club of $n$ members, where $k$ and $n$ are counting numbers with $k<n$.
\end{enumerate}
\end{prob}

\begin{teachingnote}
The following problems are optional.
\end{teachingnote}

\begin{prob}
Six coins are flipped separately (e.g., HTHHHT is different from THHHTH).  How many different results are
possible?
\end{prob}

\begin{prob}
A pizza shop always puts cheese on their pizzas.  If the shop offers
$n$ additional toppings, how many different pizzas can be ordered?
(Note: A plain cheese pizza is an option.)
\end{prob}


%\begin{prob}
%The Pig-Out restaurant offers 5 entrees, 8 side dishes, 12 desserts,
%and 6 kinds of drinks.  If you were going to select a dinner with 3
%entr\'ees, 4 side dishes, 7 desserts, and one drink, how many
%different dinners could you order?
%\end{prob}
