\newpage
\section{Decimals Aren't So Nice}\label{A:DecNotNice}


\index{paradox!1=0.9@$1 = 0.999\dots$} 

We will investigate the following question: How is $0.999\dots$
related to $1$?



\begin{prob}
What symbol do you think you should use to fill in the box below?
\[
.999\dots \,\fbox{\rule[0mm]{0mm}{2mm}\hspace{2ex}}\, 1
\]
Should you use $<$, $>$, $=$ or something else entirely?
\end{prob}


\begin{prob}
What is $1 - .999\dots$?
\end{prob}

\begin{prob}
How do you write $1/3$ in decimal notation? Express
\[
\frac{1}{3} + \frac{1}{3} + \frac{1}{3}
\]
in both fraction and decimal notation.
\end{prob}

\begin{prob}
See what happens when you follow the directions below:
\begin{enumerate}
\item Set $x = .999\dots$.
\item Compute $10x$. 
\item Compute $10x-x$.
\item From the step immediately above, what does $9x$ equal?
\item From the step immediately above, what does $x$ equal?
\end{enumerate}
\end{prob}

\begin{prob}
Are there other numbers with this weird property?
\end{prob}
