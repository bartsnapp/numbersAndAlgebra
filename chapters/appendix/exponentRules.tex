\newpage
\section{Rules of Exponents}\label{A:ExponentRules}

\fixnote{Replace with new activity here.}  

\begin{prob}
Attendance at a picnic consistently grows by 21\% each year.  The attendance this year was 5678.    
\begin{enumerate}
\item Write both a recursive and explicit representation of the relationship between the number of years and the attendance.  Use 2013 as ``year 0.''

\item Using only the attendance for 2018, how would you find the attendance for 2024 by doing only one multiplication?  What rule from school mathematics supports your solution process?
      
\item Using only the explicit form of the relationship, predict what the attendance was in 2010.  Now find the same information by only using the attendance in 2018 and one division. What rule from school mathematics supports your solution process?
     
\item What was the attendance in 2013?  What rule from school mathematics supports your solution process?
\end{enumerate}
\end{prob}
\begin{prob}
What is wrong with the following statement:  ``$5^7$ is 5 multiplied by itself 7 times.''  If the statement were true, what would $5^1$ be?  What would $5^0$ be?
\end{prob}

\begin{prob}
Why is $x^3$ not the same function as $3^x$?  We often think of multiplication as ``repeated addition,'' and we find that adding $a$ copies of $b$ gives the same result as adding $b$ copies of $a$.  Does this idea work for thinking of exponentiation as ``repeated multiplication''?  Explain.  
\end{prob}


\begin{prob}
Joe Schmo saved the King's daughter from a vicious dragon.  For such gallantry, the King offers Joe the choice of two payment plans: 
\begin{itemize}
\item Plan \#1:  \$1 today, and on subsequent days Joe will have an amount that is the cube of the day number (so, for example, on day 2, Joe  will have \$8). 
\item Plan \#2:  \$1 today, and on subsequent days Joe will have 2\% more than the previous day (for example, on Day \#2, he'll have \$1.02).  
\end{itemize}
Note that Joe must deposit the previous day's money in order to get today's money.  Assuming he wants the plan that yields the most money, which plan should he pick?  Explain your reasoning.  
\end{prob}
