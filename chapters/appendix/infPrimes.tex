\newpage
\section{There's Always Another Prime}\label{A:Pr}

We'll start off with easy questions, then move to harder ones.  

\begin{prob}
Use the Division Theorem to explain why neither $2$ nor $3$ divides
$2\cdot 3+1$.  (Hint:  Do not multiply and add.  Use the expression 
as written to reason what the quotient and remainder must be.)
\end{prob}

\begin{prob}
Use the Division Theorem to explain why neither $2$ nor $3$ nor $5$ divides
$2\cdot 3\cdot 5+1$.
\end{prob}

\begin{prob} 
Let $p_1,\dots, p_n$ be the first $n$ primes. Do any of these primes divide 
\[
p_1p_2\cdots p_n + 1?
\]
Explain your reasoning.
\end{prob}


\begin{prob} 
Suppose there were only a finite number of primes, say there were only
$n$ of them. Call them $p_1,\dots, p_n$. Could any of them divide
\[
p_1p_2\cdots p_n + 1?
\]
what does that mean? Can there really only be a finite number of
primes?
\end{prob}


\begin{prob} 
Consider the following:
\[
2\cdot 3\cdot 5\cdot 7 \cdot 11\cdot 13 + 1 = 59\cdot 509
\] 
Does this contradict our work above? If so, explain why. If not, explain
why not.
\end{prob}


