\newpage
\section{Comparative Arithmetic}\label{A:CA}

\fixnote{Need division example.  The point of the activity is that the properties that govern base-ten algorithms carry over to polynomials, but there is no carrying or borrowing.  Ultimately, we want students to see polynomials as numbers in base $x$ and to see base-ten numbers as polynomials in 10.}

\begin{prob} Compute:
\[
\begin{array}{@{}r@{}}
131\\
+122\\ \hline
\end{array}
\qquad\text{and}
\qquad
\begin{array}{@{}r@{}}
x^2+3x+1\\
+x^2+2x+2\\ \hline
\end{array}
\]
Compare, contrast, and describe your experiences.
\end{prob}

\begin{prob} Compute:
\[
\begin{array}{@{}r@{}}
139\\
+122\\ \hline
\end{array}
\qquad\text{and}
\qquad
\begin{array}{@{}r@{}}
x^2+3x+9\\
+x^2+2x+2\\ \hline
\end{array}
\]
Compare, contrast, and describe your experiences. In particular,
discuss how this is different from the first problem.
\end{prob}


\begin{prob} Compute:
\[
\begin{array}{@{}r@{}}
121\\
\times 32\\ \hline
\end{array}
\qquad\text{and}
\qquad
\begin{array}{@{}r@{}}
x^2+2x+1\\
\times~~~3x+2\\ \hline
\end{array}
\]
Compare, contrast, and describe your experiences.
\end{prob}

\begin{prob}
Expand:
\[
(x^2 + 2x + 1)(3x+2)
\]
Compare, contrast, and describe your experiences. In particular, discuss how this problem relates to the one above.
\end{prob}

\fixnote{Need division examples.}
