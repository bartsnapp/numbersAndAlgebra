\newpage
\section{Ratio Oddities}\label{A:ratioOddities}
In this activity we are going to investigate thinking about and adding
ratios.

\begin{teachingnote}
Typical answers:
  
\begin{center}
$\frac{3}{7}n+\frac{4}{7}n$, $3x+4x$, $g=\frac{4}{3}b$, $3b=4g$, $\frac{3}{4} = \frac{3x}{4x}$.
\end{center}

When writing equations, be sure to work through typical wrong answers:  (1) using letters as units versus the number of boys or girls; and (2) saying $3x=4x$ to indicate batches.  

There is no need to reduce the ratios in the answers.  Yet where appropriate, encourage answers with variables in them.

\end{teachingnote}

\begin{prob}
There are 3 boys for every 4 girls in Mrs.\ Sanders' class.
\begin{enumerate}
\item What fraction of the class are girls? 
\item List ratios that can describe this situation. 
\item If each of the number of boys and number of girls quadruples, what is the new ratio of girls to boys?
\item Write an equation relating the number of boys in the class to the number of girls in the class.

\item If the number of boys and number of girls each increase by 6, what can you say about the new ratio of boys to girls?
\end{enumerate}
\end{prob}



\begin{prob}\label{AP:C1}
Suppose the ratio of girls to boys in Smith's class is 7:3 while the
ratio of girls to boys in Jones' class is 6:5.  
\begin{enumerate}
\item If there are 50 students in Smith's class and 55 students in Jones' class, and both
classes get together for an assembly, what is the ratio of girls to
boys? Explain your reasoning.
\item What if there are 500 students in Smith's class and still 55 students in Jones' class?  
\item What if there are 5000 students in Smith's class and still 55 students in Jones' class?  
\item How do the ratios of girls to boys in the combined assembly compare to the ratios of girls to boys in the original classes?  
\item Now suppose you don't know how many students are in Smith's class and there are 55 students in Jones' class. What can you say about the ratio of girls to boys at the assembly?
\end{enumerate}
\end{prob}

\begin{prob}
Suppose you are teaching a class, and a student writes
\[
\frac{1}{4} + \frac{3}{5} = \frac{4}{9}
\]
\begin{enumerate}
\item How would you respond to this? 
\item This student is most contrary, and presents you with the following problem:
\begin{quote}
Suppose you have two cars, a 4 seater and a 5 seater. If the first car
is $1/4$ full and the second car is $3/5$ full, how full are they
together?
\end{quote}
The student then proceeds to answer their question with ``The answer
is $4/9$.'' How do you address this?

\begin{teachingnote}
You might want to ask what happens if the first car is $1/4$ full and $6/10$ full;  or suggest the first car is 2/8 full, because 2/8 = 1/4.
\end{teachingnote}

\item This student's reasoning suggest a new kind of ``addition'' of ratios.  Let's use $\oplus$ for this new form of ``addition.'' So
\[
\frac{a}{b} \oplus \frac{c}{d} = \frac{a+c}{b+d}.
\]
For which of the previous problems is does this ``addition'' give the correct answer?  What is going on?  
\item Use the student's context of seats and cars to reason about how $\frac{a}{b} \oplus \frac{c}{d}$ compares with $\frac{a}{b}$ and $\frac{c}{d}.$
\end{enumerate}
\end{prob}

\begin{prob} 
Let's think a bit more about $\oplus$. If you were going to plot
$\frac{a}{b}$ and $\frac{c}{d}$ on a number line, what can you 
say about the location of 
$\frac{a}{b} \oplus \frac{c}{d}$? Is this always the case, or does it
depend on the values of $a$, $b$, $c$, and $d$?  
\emph{Hint}:  Assume that all of the letters are positive.  Use specific numbers and a 
context; then try to reason generally.
\end{prob}

\begin{teachingnote}
Here you will probably not only want to have the students realize that
$\frac{a}{b} \oplus \frac{c}{d}$ is between both $\frac{a}{b}$ and
$\frac{c}{d}$, but that the location varies by which denominator is
largest.

Another approach is to compare slopes of vectors $(b,a)$, $(d,c)$, and $(b+d,a+c)$, all originating at the origin.  Through specific examples, students can reason that the vector sum (and therefore its slope) is between the others.  
\end{teachingnote}

%
%\begin{prob}
%Again, suppose the ratio of girls to boys in Smith's class is 7:3
%while the ratio of girls to boys in Jones' class is 6:5.  If there are
%40 students in Smith's class and 55 students in Jones' class, and both
%classes get together for an assembly, what is the ratio of girls to
%boys at the assembly? Explain your reasoning.
%\end{prob}







