\newpage
\section{Ratio Oddities}\label{A:ratioOddities}

In this activity we are going to investigate thinking about and adding
ratios.

\fixnote{Need units rates and to distinguish ratios and rates.  Maybe distinguish part/part to part/whole.}

\begin{prob}
There are 3 boys for every 4 girls in Mrs.\ Sanders' class.
\begin{enumerate}
\item What fraction of the class are girls? 
\item List ratios that can describe this situation. 
\item If each of the number of boys and number of girls quadruples, what is the new ratio of girls to boys?
\item Write an equation relating the number of boys in the class to the number of girls in the class.
\item If the number of boys and number of girls each increase by 6, what can you say about the new ratio of boys to girls?
\end{enumerate}
\end{prob}

\begin{prob}\label{AP:C1}
Suppose the ratio of girls to boys in Smith's class is 7:3 while the
ratio of girls to boys in Jones' class is 6:5.  
\begin{enumerate}
\item If there are 50 students in Smith's class and 55 students in Jones' class, and both
classes get together for an assembly, what is the ratio of girls to
boys? Explain your reasoning.
\item If there are 500 students in Smith's class and 55 students in Jones' class, and both
classes get together for an assembly, what is the ratio of girls to
boys? Explain your reasoning.
\item If there are 5000 students in Smith's class and 55 students in Jones' class, and both
classes get together for an assembly, what is the ratio of girls to
boys? Explain your reasoning.
\item Now suppose you don't know how many students are in Smith's class and there are 55 students in Jones' class. What is the best answer you can give for the ratio of girls to boys at the assembly?
\fixnote{Perhaps allow an answer with a variable in it.}  
\end{enumerate}
\end{prob}


\fixnote{In the following problem, include the first car as 2/8 full, because 2/8 = 1/4.}
\begin{prob}
Suppose you are teaching a class, and a student writes
\[
\frac{1}{4} + \frac{3}{5} = \frac{4}{9}
\]
\begin{enumerate}
\item How would you respond to this? 
\item This student is most contrary, and presents you with the following problem:
\begin{quote}
Suppose you have two cars, a 4 seater and a 5 seater. If the first car
is $1/4$ full and the second car is $3/5$ full, how full are they
together?
\end{quote}
The student then proceeds to answer their question with ``The answer
is $4/9$.'' How do you address this?
\end{enumerate}
\begin{teachingnote}
You might want to ask what happens if the first car is $1/4$ full and $6/10$ full.
\end{teachingnote}
\end{prob}



\begin{prob}\label{AP:C2}
Again, suppose the ratio of girls to boys in Smith's class is 7:3
while the ratio of girls to boys in Jones' class is 6:5.  If there are
40 students in Smith's class and 55 students in Jones' class, and both
classes get together for an assembly, what is the ratio of girls to
boys at the assembly? Explain your reasoning.
\end{prob}


\begin{prob} 
Let's use $\oplus$ for this new form of ``addition.'' So
\[
\frac{a}{b} \oplus \frac{c}{d} = \frac{a+c}{b+d}.
\]
I claim that in Problem~\ref{AP:C1} we could solve using
\[
\frac{a}{b} \oplus \frac{c}{d} = \frac{a+c}{b+d}.
\]
However, in Problem~\ref{AP:C2} we could not. What is going on here?
\end{prob}



\begin{prob} 
Let's think a bit more about $\oplus$. If you were going to plot
$\frac{a}{b}$ and $\frac{c}{d}$ on a number line, where is
$\frac{a}{b} \oplus \frac{c}{d}$? Is this always the case, or does it
depend on the values of $a$, $b$, $c$, and $d$? You should give an
explanation based on context, and an explanation based on algebra.
\end{prob}

\begin{teachingnote}
Here you will probably not only want to have the students realize that
$\frac{a}{b} \oplus \frac{c}{d}$ is between both $\frac{a}{b}$ and
$\frac{c}{d}$, but that the location varies by which denominator is
largest.
\end{teachingnote}



