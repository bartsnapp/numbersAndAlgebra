\newpage
\section{Shampoo, Rinse, \dots}\label{A:Shampoo}
\index{repeating decimal}
\index{terminating decimal}

\fixnote{Perhaps look for pairs:  1/8 and 1/12;  1/11 and 1/9.}  

We're going to investigate the following question: If $a$ and $b$ are
integers with $b \ne 0$, what can you say about the decimal
representation of $a/b$? Let's see if we can get to the bottom of
this.

\begin{prob}\label{AR:exp}
Write the following fractions in decimal notation. Which have a
``terminating'' and which have a ``non-terminating'' decimal?
\[
\begin{array}{cccccccccccc}
 \dfrac{1}{2}, &  \dfrac{1}{3}, &  \dfrac{1}{4}, &  \dfrac{1}{5}, &  \dfrac{1}{6}, & \dfrac{1}{8}, & \dfrac{1}{9}, &  \dfrac{1}{10}, &  \dfrac{1}{11}, &  \dfrac{1}{12}, &  \dfrac{1}{13},  & \dfrac{1}{15}\\ \\
 \dfrac{1}{16}, & \dfrac{1}{20}, &  \dfrac{1}{24}, &  \dfrac{1}{25}, &  \dfrac{1}{28}, &  \dfrac{1}{32}, &  \dfrac{1}{35}, &  \dfrac{1}{40}, &  \dfrac{1}{42}, &  \dfrac{1}{48}, &  \dfrac{1}{64}, &  \dfrac{1}{80}.
\end{array}
\]
\end{prob}

\begin{prob}\label{AR:conj}
Can you find a pattern from your results from Problem~\ref{AR:exp}?
Use your pattern to guess whether the following fractions
``terminate''?  
\[
\dfrac{1}{61}\qquad \dfrac{1}{625} \qquad \dfrac{1}{6251}
\]
\end{prob}


\begin{prob}
Can you explain why your conjecture from Problem~\ref{AR:conj} is true?
\end{prob}

\begin{prob}\label{AR:comp} Use long division to compute $1/7$.
\end{prob}

\begin{prob}\index{Division Theorem!for integers}
State the Division Theorem for integers.
\end{prob}

\begin{prob}\label{AR:div}
Considering the solution of Problem~\ref{AR:comp}, explicitly
explain how the Division Theorem for integers appears in your work.
\end{prob}


\begin{prob} 
In each instance of the Division Theorem in Problem~\ref{AR:div}, what
is the divisor? What does this say about the remainder?
\end{prob}

\begin{prob} 
What can you say about the decimal representation of $a/b$ when $a$
and $b$ are integers with $b\ne 0$?
\end{prob}

\begin{prob} 
Assuming that the pattern holds, is the number
\[
.123456789101112131415161718192021\dots
\]
a rational number? Explain your reasoning.
\end{prob}


\begin{prob}\label{AR:nines} 
Compute $\frac{1}{9}$, $\frac{1}{99}$, and $\frac{1}{999}$. Can you
find a pattern? Can you explain why your pattern holds?
\end{prob}


\begin{prob}
Use your work from Problem~\ref{AR:nines} to give the fraction form of
the following decimals:
\begin{enumerate}
\item $0.\overline{357}$
\item $23.\overline{459}$
\item $0.23\overline{4598}$
\item $76.3\overline{421}$
\end{enumerate}
\end{prob}

