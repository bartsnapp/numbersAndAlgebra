\newpage
\section{Flour Power}\label{A:FlourPower}


\begin{prob} 
Suppose a cookie recipe calls for $2$ cups of flour. If you have $6$
cups of flour total, how many batches of cookies can you make?
\begin{enumerate}
\item Numerically solve the problem.
\item Draw a picture representing the situation or describe actions with objects a student could carry out to solve the problem.
\item Identify whether the problem is asking ``How many groups?'' or ``How many in each group?'' or something else entirely.
\end{enumerate}
\end{prob}


\begin{prob} 
You decide that $2$ cups of flour per batch is too much for your
taste---you think you'll try $1\frac{1}{2}$ cups per batch. If you
have $6$ cups of flour, how many batches of cookies can you make?
\begin{enumerate}
\item Numerically solve the problem.
\item\label{AP:b} Draw a picture representing the situation or
  describe actions with objects a student could carry out to solve the
  problem.
\item Identify whether the problem is asking ``How many groups?'' or ``How many in each group?'' or something else entirely.
\end{enumerate}
\end{prob}

\begin{prob}
Somebody once told you that ``to divide fractions, you invert and
multiply.'' Discuss how this rule is manifested in part \ref{AP:b} of
the problem above.
\end{prob}



\begin{prob} 
You have $2$ snazzy stainless steel containers, which hold a total of
$6$ cups of flour. How many cups of flour does $1$ container hold?
\begin{enumerate}
\item Numerically solve the problem.
\item Draw a picture representing the situation or describe actions with objects a student could carry out to solve the problem.
\item Identify whether the problem is asking ``How many groups?'' or ``How many in each group?'' or something else entirely.
\end{enumerate}
\end{prob}


\begin{prob} 
Now you have $3$ beautiful decorative bowls, which hold a total of
$1/2$ cup of flour. How many cups of flour does $1$ decorative bowl
hold?
\begin{enumerate}
\item Numerically solve the problem.
\item Draw a picture representing the situation or describe actions with objects a student could carry out to solve the problem.
\item Identify whether the problem is asking ``How many groups?'' or ``How many in each group?'' or something else entirely.
\end{enumerate}
\end{prob}

\begin{prob}
Somebody once told you that ``to divide fractions, you invert and
multiply.'' Discuss how this rule is manifested in part \ref{AP:b} of
the problem above.
\end{prob}
