\newpage
\section{Flour Power}\label{A:FlourPower}

\begin{prob} 
Suppose a cookie recipe calls for $2$ cups of flour. If you have $6$
cups of flour total, how many batches of cookies can you make?
\begin{enumerate}
\item Draw a picture representing the situation, and use pictures to solve the problem.
\item Identify whether the problem is asking ``How many groups?'' or ``How many in one group?'' or something else entirely.
\item You find another recipe that calls for $1\frac{1}{2}$ cups per batch. If you have $6$ cups of flour, how many batches of these cookies can you make?  Again use pictures to solve the problem.
\item Somebody once told you that ``to divide fractions, you invert and
multiply.'' Discuss how this rule is manifested in this problem.
\end{enumerate}
\end{prob}

\begin{prob} 
You have $2$ snazzy stainless steel containers (both the same size), which hold a total of
$6$ cups of flour. How many cups of flour does $1$ container hold?
\begin{enumerate}
\item Draw a picture representing the situation, and use pictures to solve the problem.
\item Identify whether the problem is asking ``How many groups?'' or ``How many in one group?'' or something else entirely.
\item It turned out that the 6 cups of flour filled exactly $1\frac{1}{2}$ of your containers.  How many cups of flour does $1$ container hold?  Again use pictures to solve the problem.
\item Somebody once told you that ``to divide fractions, you invert and
multiply.'' Discuss how this rule is manifested in this problem.
\end{enumerate}
\end{prob}


%\begin{prob} 
%Now you have $3$ beautiful decorative bowls, which hold a total of
%$1/2$ cup of flour. How many cups of flour does $1$ decorative bowl
%hold?
%\begin{enumerate}
%\item Draw a picture representing the situation, and use your picture to solve the problem.
%\item Identify whether the problem is asking ``How many groups?'' or ``How many in one group?'' or something else entirely.
%\item Somebody once told you that ``to divide fractions, you invert and
%multiply.'' Discuss how this rule is manifested in this problem.
%\end{enumerate}
%\end{prob}
%
