\newpage
\section{Picture Yourself Dividing}


We want to understand how to visualize 
\[
\frac{a}{b} \div \frac{c}{d}
\]
Let's see if we can ease into this like a cold swimming pool.

\begin{prob}
Draw a picture that shows how to compute:
\[
10\div 5
\]
Explain how your picture could be redrawn for other similar
numbers. Write two story problems solved by this expression, one
asking for ``how many groups'' and the other asking for ``how many in
one group.''
\end{prob}

\begin{prob}
Try to use a similar process to the one you used in the first problem
to draw a picture that shows how to compute:
\[
\frac{1}{4} \div 3
\]
Explain how your picture could be redrawn for other similar numbers.
Write two story problems solved by this expression, one asking for
``how many groups'' and the other asking for ``how many in one
group.''
\end{prob}


\begin{prob}
Try to use a similar process to the one you used in the first two problems
to draw a picture that shows how to compute:
\[
3 \div \frac{1}{4}
\]
Explain how your picture could be redrawn for other similar numbers.
Write two story problems solved by this expression, one asking for
``how many groups'' and the other asking for ``how many in one
group.''
\end{prob}

\fixnote{Also incorporate $1\frac{3}{4}\div \frac{1}{2}$}

\begin{prob}
Try to use a similar process to the one you used in the first three problems
to draw a picture that shows how to compute:
\[
\frac{7}{5} \div \frac{3}{4}
\]
Explain how your picture could be redrawn for other similar numbers.
Write two story problems solved by this expression, one asking for
``how many groups'' and the other asking for ``how many in each
group.''
\end{prob}

\begin{prob}
Explain how to draw pictures to visualize:
\[
\frac{a}{b} \div \frac{c}{d}
\]
\end{prob}

\begin{prob}
Use pictures to explain why:
\[
\frac{a}{b} \div \frac{c}{d} = \frac{a}{b} \cdot \frac{d}{c}
\]
\end{prob}
