\newpage
\section{Complete Squares}\label{A:completeSquares}

\begin{prob}
In the following list of equations, solve those that are \textbf{easy} to solve.  
\begin{enumerate}
\item $x^2=5$
\item $x^2 - 4 = 2$
\item $x^2 - 4x = 2$
\item $2x^2=1$
\item $(x-2)^2=5$
\end{enumerate}
\end{prob}
\vspace{0.3in}

\begin{prob}
Regarding the previous problem, state the property of numbers that made all but one of the equations easy to solve.  
\begin{teachingnote}
If $u^2=a$ then $u=\pm\sqrt{a}$.  Be sure to spend some time discussing why 
$$\pm\sqrt{\frac{1}{2}}=\pm\frac{1}{\sqrt{2}}=\pm\frac{\sqrt{2}}{2}.$$
\end{teachingnote}
\vspace{0.3in}
\end{prob}

\begin{prob}
Although $160$ is not a square in base ten, what could you add to $160$ so that the result would be a square number?  
\end{prob}

\begin{prob}
 Consider the polynomial expression $x^2+6x$ to be a number in base $x$.  We want to add to this polynomial so that the result is a square in base $x$.  
\begin{enumerate}
\item Use ``flats'' and ``longs'' to draw a picture of this polynomial as a number in base $x$, adding enough ``ones'' so that you can arrange the polynomial into a square.  
\vspace{0.5in}
\item What ``feature'' of the square does the new polynomial expression represent?  
\item Why does it make sense to call this technique ``completing the square''? 
\item Use your picture to help you solve the equation $x^2+6x=5$.  
\end{enumerate}
\end{prob}


\begin{prob}
Complete the square to solve the following equations: 
\begin{enumerate}
\item $x^2+3x=4$
\vspace{.8in}
\item $x^2+bx=q$
\vspace{.8in}
\item $2x^2+8x=12$
\vspace{.8in}
\item $ax^2+bx+c=0$
\vspace{.8in}
\end{enumerate}  
\end{prob}

\begin{prob}
Solve the following equation
\[
x^5 - 4x^4 - 18x^3 + 64x^2 + 17x -60 = 0
\]
assuming you know that $1$, $-1$, and $3$ are roots.
\end{prob}

