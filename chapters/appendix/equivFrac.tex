\newpage
\section{Picture Models for Equivalent Fractions}\label{A:EF}

\begin{teachingnote}

Step 1.  Use the first problem (use paper to show 3/8) to generate the meaning of fraction from the Common Core State Standards:  

\begin{quote}
3.NF.1. Understand a fraction $1/b$ as the quantity formed by $1$ part when a
whole is partitioned into $b$ equal parts; understand a fraction $a/b$ as
the quantity formed by $a$ parts of size $1/b$.

Source:  \url{http://www.corestandards.org/Math/Content/3/NF/A/1/}
\end{quote}


The code 3.NF.1 means ``third grade, number and operations--fractions, standard 1.''  These standards are written to be read by teachers, not students.

Step 2.  Introduce the formal definition of rational number and the set of rational numbers, as in the beginning of section 2.4.  

A rational number can be represented as $a/b$ with integers $a$ and $b$, where $b$ is not $0$.  

Distinguish fraction (a representation) from rational number, highlighting the phrase ``can be'' in the definition.  Have students generate fractions that are not rational numbers as well as rational numbers not represented as fractions.  

Introduce the letter $\Q$ (usually in ``black-board bold'' font) to denote the set of all rational numbers. 

Step 3.  Complete Activity A.16.  The point is to use the meaning of fractions above to explain why fractions are equivalent.  And the approach is ``reasoning generally with specific numbers.'' 

A.16.2.  For $2/3 = 4/6$, some students will be tempted to draw 2/3, draw 4/6 and then say, ``See!''  With this method, it is not clear why the pieces should line up.  Much better to use 2/3 to create 4/6 by cutting each of the thirds into two equal pieces.  

A.16.3.  For $3/6 = 2/4$, some students will be tempted to say ``Because they both equal 1/2.''  To explain why the pieces will have to line up, it is clearer (and more general) to go through a common denominator, such as 12ths or 24ths.  
 
A.16.4.  To show that $a/b = c/d$, generalize the approach from the previous problem:  Thinking of the common denominator $bd$, cut the $a/b$ parts each into $d$ parts.  Then we have $ad$ parts of size $1/(bd)$.  Cut the $c/d$ parts each into $b$ parts.  Then we have $cb$ parts of size $1/(db)$.  For the two fractions to be equal, the $ad$ parts of size $1/(bd)$ must be equal to the $cb$ parts of size $1/(db)$.  

A.16.5.  In the picture from A.16.4, because the parts are the same size (i.e., $1/(bd)$), it must follow that $ad = bc$.  (Argue both directions:  if the fractions are equal, then $ad = bc$;  if $ad = bc$, then the fractions must be equal.)  
\end{teachingnote}

\begin{prob}
Get out a piece of paper and show $\dfrac{3}{8}$.  Explain how you know.  
\end{prob}

\begin{prob} 
Draw pictures to explain why:
\[
\frac{2}{3} = \frac{4}{6}
\]
Explain how your pictures show this.
\end{prob}


\begin{prob} 
Draw pictures to explain why:
\[
\frac{3}{6} = \frac{2}{4}
\]
Explain how your pictures show this.
\end{prob}



\begin{prob} 
Given equivalent fractions with $0< a\le b$ and $0 < c\le d$:
\[
\frac{a}{b} = \frac{c}{d}
\]
Give a procedure for representing this equation with pictures.
\end{prob}


\begin{prob} 
Explain, without cross-multiplication, why if $0< a\le b$ and $0 < c\le d$:
\[
\frac{a}{b} = \frac{c}{d}\qquad \text{if and only if}\qquad ad = bc
\]
Feel free to use pictures as part of your explanation.
\end{prob}
